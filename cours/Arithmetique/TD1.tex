\documentclass[a4paper,12pt]{article}
\usepackage[utf8]{inputenc}
\usepackage[T1]{fontenc}
\usepackage[french]{babel}
\usepackage{geometry}
\usepackage{fancyhdr}

\geometry{margin=2cm}
\setlength{\headheight}{15pt}


\pagestyle{fancy}
\fancyhf{}
\fancyhead[L]{TD1 Arithmétique}
\fancyhead[C]{Mignard Mael}
\fancyhead[R]{\today}
\fancyfoot[C]{\thepage}
\fancyfoot[R]{\LaTeX}

\title{TD1 Arithmétique}
\author{Mignard Mael}
\date{\today}

\begin{document}
\maketitle
\thispagestyle{empty}

\section{Exercice 1}
\subsection{Question a}
\begin{itemize}
    \item $2^6 = 64$
    \item $(-3)^4 = 81$
    \item $5^0 = 1$
    \item $(-2)^5 = -32$
\end{itemize}

\subsection{Question b}
\begin{itemize}
    \item $2^3 \times 2^5 = 2^{3+5} = 2^8 = 256$
    \item $\frac{7^8}{7^3} = 7^{8-3} = 7^5 = 16807$
    \item $(3^2)^4 = 3^{2 \times 4} = 3^8 = 6561$
\end{itemize}

\subsection{Question c}
\begin{itemize}
    \item $10 \times 10 \times 10 = 10^3 = 1000$
\end{itemize}

\clearpage
\section{Exercice 2}
\subsection{Question a}
\begin{itemize}
    \item Si $n$ est pair : $n = 2 \times k$
    \item Alors $n^2 = (2 \times k)^2 = 4 \times k^2 = 2 \times (2 \times k^2)$
    \item On nomme $K = 2 \times k^2$
    \item Donc $n^2 = 2 \times K$, donc $n^2$ est pair
\end{itemize}

\subsection{Question b}
\begin{itemize}
    \item Si $n$ est impair : $n = 2 \times k + 1$
    \item Alors $n^2 = (2 \times k + 1)^2 = 4 \times k^2 + 4 \times k + 1 = 2 \times (2 \times k^2 + 2 \times k) + 1$
    \item On nomme $K = 2 \times k^2 + 2 \times k$
    \item Donc $n^2 = 2 \times K + 1$, donc $n^2$ est impair
\end{itemize}

\subsection{Question c}
\begin{itemize}
    \item Vérifier que $(ab)^n = a^n \times b^n$
    \item $(ab)^n = ab \times ab \times \dots \times ab$ (n fois)
    \item $= a \times a \times \dots \times a$ (n fois) $\times b \times b \times \dots \times b$ (n fois)
    \item $= a^n \times b^n$
\end{itemize}

\clearpage
\section{Exercice 3}
\subsection{Question a}
\subsubsection{Décomposition en facteurs premiers}
\begin{itemize}
    \item $84 = 2 \times 42 = 2 \times 2 \times 21 = 2^2 \times 3 \times 7$
    \item $210 = 2 \times 105 = 2 \times 3 \times 35 = 2 \times 3 \times 5 \times 7$
\end{itemize}

\subsubsection{Plus grand commun diviseur}
\begin{itemize}
    \item $pgcd(84,210) = 2 \times 3 \times 7 = 42$
\end{itemize}

\subsection{Question b}
\subsubsection{Décomposition en facteurs premiers}
\begin{itemize}
    \item $144 = 2^4 \times 3^2$
    \item $198 = 2 \times 3^2 \times 11$
\end{itemize}

\subsubsection{Plus grand commun diviseur}
\begin{itemize}
    \item $pgcd(144,198) = 2 \times 3^2 = 18$
\end{itemize}

\subsection{Question c}
\subsubsection{Décomposition en facteurs premiers}
\begin{itemize}
    \item $128 = 2^7$
    \item $160 = 2^5 \times 5$
\end{itemize}

\subsubsection{Plus grand commun diviseur}
\begin{itemize}
    \item $pgcd(128,160) = 2^5 = 32$
\end{itemize}

\section{Exercice 5}
\subsection{Question a}
\begin{itemize}
    \item Décomposer en facteurs premiers :
    \begin{itemize}
        \item $84 = 2^2 \times 3 \times 7$
        \item $126 = 2 \times 3^2 \times 7$
        \item $210 = 2 \times 3 \times 5 \times 7$
    \end{itemize}
    \item PGCD des 3 nombres : $pgcd(84,126,210) = 2^1 \times 3^1 \times 7^1 = 42$
    \item La longueur d'un morceau est donc de 42 cm
\end{itemize}

\end{document}
