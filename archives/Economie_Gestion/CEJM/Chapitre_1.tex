\documentclass[12pt,a4paper]{article}
\usepackage[utf8]{inputenc}
\usepackage[T1]{fontenc}
\usepackage[french]{babel}
\usepackage{geometry}
\usepackage{amsmath,amssymb}
\usepackage{graphicx}
\usepackage{array}
\usepackage{tabularx}
\usepackage{booktabs}
\usepackage{xcolor}
\usepackage{fancyhdr}
\usepackage{hyperref}

% Mise en page
\geometry{margin=2.2cm}
\pagestyle{fancy}
\fancyhf{}
\fancyhead[L]{CEJM — Thème 1}
\fancyhead[R]{Chapitre 1}
\fancyfoot[C]{\thepage}
\setlength{\headheight}{15pt}

% Couleurs simples
\definecolor{bleu}{RGB}{0,90,170}
\definecolor{gris}{RGB}{245,245,245}

% Encadrés simples sans packages additionnels
\newcommand{\Encadre}[2][bleu]{%
  \noindent\fcolorbox{#1}{gris}{\parbox{\dimexpr\textwidth-2\fboxsep-2\fboxrule\relax}{#2}}
}

% Titre
\title{\textbf{CEJM} \\[3mm]\large Thème 1 : L'intégration de l'entreprise dans son environnement\\[2mm]\normalsize Chapitre 1 : Les agents économiques en relation avec l'entreprise}
\author{}
\date{\vspace{-6mm}}

\begin{document}
\maketitle
\vspace{-8mm}
\Encadre{\textbf{Problématique :} Comment s'établissent les relations entre l'entreprise et son environnement économique ?}

\tableofcontents
\newpage

\section{Synthèse}
L'activité économique est l'activité humaine qui organise la production, l'échange et la consommation de biens et de services pour satisfaire des besoins.\\
Pour mener leurs activités, les entreprises sont en relation avec de nombreux agents économiques avec lesquels elles effectuent des échanges de diverses natures.

\section{Les acteurs de l'activité économique}
En économie, on parle \textbf{d'agents économiques} (ou secteurs institutionnels). Un agent économique est un individu ou un ensemble d'individus constituant un centre de décision économique indépendant.\\
L'analyse regroupe ces centres de décision en catégories selon leur fonction principale.

\subsection{Catégories principales d'agents}
\begin{itemize}
  \item \textbf{Ménages} : individus ou groupes vivant dans un même logement. Fonction principale : \textit{consommation} de biens et services pour satisfaire directement leurs besoins.
  \item \textbf{Entreprises} : produisent des biens et/ou des services pour les vendre sur un marché afin de réaliser un \textit{profit}.
  \item \textbf{Banques} : assurent le \textit{financement de l'économie} par la collecte et l'octroi de crédits.
  \item \textbf{Administrations publiques} (État, Sécurité sociale, collectivités locales) : répondent à des besoins d'intérêt général et produisent des \textit{services non marchands}.
\end{itemize}

\subsection{Production marchande et non marchande}
\begin{description}
  \item[Production marchande] assurée par les entreprises : biens et services vendus sur un marché à un prix concurrentiel.
  \item[Production non marchande] assurée par les administrations publiques : fournie gratuitement ou à des prix non significatifs (\textless{} 50\% du coût de production).
\end{description}

\section{Interactions sur les marchés}
La réalité peut être représentée par un \textit{circuit économique}, modèle schématique des échanges et interdépendances entre agents économiques.\\
\Encadre{\textbf{Vidéo :} \url{https://www.citeco.fr/le-circuit-\%C3\%A9conomique}}

\section{Relations de l'entreprise avec les ménages}
\subsection{Marché des biens et services}
\begin{itemize}
  \item \textbf{Flux réels} des entreprises vers les ménages : biens et services.
  \item \textbf{Flux monétaires} des ménages vers les entreprises : paiements (chiffre d'affaires).
\end{itemize}
Les décisions de consommation dépendent des préférences, du budget et de la \textit{conjoncture} (moral des ménages). Une conjoncture favorable stimule la consommation ; défavorable, elle incite à l'épargne. Les entreprises doivent adapter leur approche client.\\
Les comportements évoluent avec le numérique : nécessité d'anticiper et de personnaliser la relation avec des consommateurs plus nomades et volatils.

\subsection{Marché du travail}
Rencontre entre ménages et entreprises : \textbf{travail} contre \textbf{salaire}.\\
\begin{itemize}
  \item \textbf{Flux réels} : quantité de travail fournie.
  \item \textbf{Flux monétaires} : salaires versés aux ménages.
\end{itemize}

\section{Relations avec les autres entreprises}
Fonction principale de l'entreprise : \textbf{produire}. Elle achète à d'autres entreprises (fournisseurs), générant échanges B2B.
\begin{itemize}
  \item Fournisseurs de \textit{biens} : matières premières, produits intermédiaires...
  \item Fournisseurs de \textit{services} : transport, entretien, publicité...
  \item Fournisseurs d'\textit{immobilisations} : biens de production durables (mobilier, outils...).
\end{itemize}
Ces relations croisent flux réels et flux monétaires. Une \textbf{collaboration réussie} avec fournisseurs et partenaires améliore la performance et la création de valeur.

\section{Relations avec les banques}
Le \textbf{marché des capitaux} met en relation offreurs (épargnants) et demandeurs de capitaux (entreprises, etc.). Le \textbf{prix du capital} est le \textit{taux d'intérêt}.\\
\textit{Financement externe indirect} : la banque joue l'intermédiaire entre agents économiques.

\subsection*{Focus : le marché financier}
Lieu d'émission et d'échange des \textit{valeurs mobilières} (actions, obligations). \textit{Financement externe direct} : rencontre directe via des titres.

\begin{tabularx}{\textwidth}{@{}p{0.48\textwidth} p{0.48\textwidth}@{}}
\multicolumn{1}{c}{\textbf{Marché primaire}} & \multicolumn{1}{c}{\textbf{Marché secondaire}} \\
\midrule
Émission de nouvelles valeurs (marché du neuf).\\
Finance entreprises, État, collectivités (investissements). & Marché de l'occasion.\\
Échange et cotation de titres déjà émis. \\
\end{tabularx}

\vspace{2mm}
\Encadre{\textbf{Vidéo :} \url{https://www.youtube.com/watch?v=MQUuCmO98DE}}

\paragraph{Compléments — Pourquoi un marché financier ?}
\begin{itemize}
  \item Accélérateur de croissance : capitaux importants pour se développer.
  \item Levées de fonds moins coûteuses et plus simples (moins d'intermédiation).
  \item Transparence financière accrue, appréciée par les financeurs et partenaires.
  \item Notoriété renforcée via la diffusion d'information.
\end{itemize}

\section{Relations avec l'État}
L'État fournit des \textbf{services non marchands} (éducation, sécurité, justice, infrastructures...). Il verse des \textbf{prestations sociales} (allocations, aides, indemnités chômage) et attribue des \textbf{subventions} à certaines entreprises. Ces dépenses sont financées par \textbf{impôts}, \textbf{taxes} et \textbf{cotisations sociales}.\\
\textit{Exemple} : \textbf{Bpifrance} (banque publique d'investissement) finance et accompagne les projets d'entreprises (innovation, international, fonds propres, reprise...).\\
Les entreprises contribuent au financement public via l'\textit{impôt sur les sociétés}, les \textit{cotisations sociales}, etc.

\section*{Mots-clés}
Agents économiques — Ménages — Entreprises — Administrations publiques — Banques — Production marchande — Production non marchande — Marché des biens et services — Marché du travail — Marché des capitaux — Marché financier.

\end{document}
