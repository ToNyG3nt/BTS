\documentclass[12pt,a4paper]{article}
\usepackage[utf8]{inputenc}
\usepackage[french]{babel}
\usepackage[T1]{fontenc}
\usepackage{geometry}
\usepackage{fancyhdr}
\usepackage{listings}
\usepackage{xcolor}

% Configuration de la page
\geometry{left=2cm, right=2cm, top=2.5cm, bottom=2cm}
\setlength{\headheight}{15pt}

% Configuration des en-têtes et pieds de page
\pagestyle{fancy}
\fancyhf{}
\fancyhead[L]{Gestion du Patrimoine Informatique}
\fancyhead[R]{BTS SIO}
\fancyfoot[C]{\thepage}

% Configuration pour le code
\lstdefinestyle{mystyle}{
    backgroundcolor=\color{gray!10},
    basicstyle=\ttfamily\scriptsize,
    breaklines=true,
    frame=single,
    numbers=left,
    numberstyle=\tiny,
    showstringspaces=false
}
\lstset{style=mystyle}

\title{
    \Large{\textbf{Installation GLPI 10.0.20}} \\
    \large{Arch Linux + Apache + MariaDB}
}

\author{\textbf{BTS SIO} - Gestion du Patrimoine Informatique}
\date{\today}

\begin{document}

\maketitle
\thispagestyle{empty}

% ========================================
% INTRODUCTION
% ========================================
\section{Configuration}

\textbf{Environnement :}
\begin{itemize}
    \item Système : Arch Linux
    \item Serveur web : Apache (httpd)
    \item Base de données : MariaDB
    \item GLPI : Version 10.0.20
    \item URL : http://10.100.133.85/glpi/index.php?noAUTO=1
\end{itemize}

% ========================================
% INSTALLATION
% ========================================
\section{Installation LAMP}

\subsection{Installation des paquets}
\begin{lstlisting}[language=bash]
# Mise à jour et installation
sudo pacman -Syu
sudo pacman -S apache mariadb php php-apache php-gd php-intl php-zip php-curl php-mbstring php-xml php-mysql

# Démarrage des services
sudo systemctl start httpd mariadb
sudo systemctl enable httpd mariadb
\end{lstlisting}

\subsection{Configuration MariaDB}
\begin{lstlisting}[language=bash]
# Initialisation
sudo mysql_install_db --user=mysql --basedir=/usr --datadir=/var/lib/mysql
sudo mysql_secure_installation

# Création base GLPI
sudo mysql -u root -p
\end{lstlisting}

\begin{lstlisting}[language=sql]
CREATE DATABASE glpidb CHARACTER SET utf8mb4 COLLATE utf8mb4_unicode_ci;
CREATE USER 'glpiuser'@'localhost' IDENTIFIED BY 'MotDePasseSecurise123!';
GRANT ALL PRIVILEGES ON glpidb.* TO 'glpiuser'@'localhost';
FLUSH PRIVILEGES;
EXIT;
\end{lstlisting}

\subsection{Configuration Apache}
\begin{lstlisting}[language=bash]
# Edition httpd.conf
sudo nano /etc/httpd/conf/httpd.conf
# Décommenter : LoadModule rewrite_module modules/mod_rewrite.so
# Ajouter : LoadModule php_module modules/libphp.so
# Ajouter : AddHandler php-script .php
# Ajouter : Include conf/extra/php_module.conf

# Créer php_module.conf
sudo nano /etc/httpd/conf/extra/php_module.conf
\end{lstlisting}

Contenu de php\_module.conf :
\begin{lstlisting}
<IfModule php_module>
    DirectoryIndex index.php index.html
    <FilesMatch "\.php$">
        SetHandler application/x-httpd-php
    </FilesMatch>
</IfModule>
\end{lstlisting}

\begin{lstlisting}[language=bash]
# Test et redémarrage
sudo httpd -t
sudo systemctl restart httpd
\end{lstlisting}

% ========================================
% INSTALLATION GLPI
% ========================================
\section{Installation GLPI}

\subsection{Téléchargement et installation}
\begin{lstlisting}[language=bash]
# Téléchargement GLPI 10.0.20
cd /tmp
wget https://github.com/glpi-project/glpi/releases/download/10.0.20/glpi-10.0.20.tgz
tar -xzf glpi-10.0.20.tgz

# Installation
sudo mv glpi /srv/http/
sudo chown -R http:http /srv/http/glpi/
sudo chmod -R 755 /srv/http/glpi/
sudo chmod -R 775 /srv/http/glpi/files/ /srv/http/glpi/config/
\end{lstlisting}

\subsection{Installation web}
\begin{enumerate}
    \item Accéder à : \textbf{http://10.100.133.85/glpi/index.php?noAUTO=1}
    \item Choisir \textbf{Français} → OK
    \item Accepter la licence GPL → Continuer
    \item Vérifier les prérequis (tous en vert) → Continuer
    \item Configuration BDD :
    \begin{itemize}
        \item Serveur : localhost
        \item Utilisateur : glpiuser
        \item Mot de passe : MotDePasseSecurise123!
        \item Base : glpidb
    \end{itemize}
    \item Attendre l'initialisation → Continuer
    \item Utiliser GLPI
\end{enumerate}

\subsection{Première connexion}
\begin{enumerate}
    \item Se connecter avec : \textbf{glpi / glpi}
    \item Changer le mot de passe obligatoirement
    \item Supprimer le répertoire d'installation :
\end{enumerate}

\begin{lstlisting}[language=bash]
sudo rm -rf /srv/http/glpi/install/
\end{lstlisting}

% ========================================
% CONFIGURATION
% ========================================
\section{Configuration des utilisateurs}

\subsection{Comptes par défaut}
\begin{itemize}
    \item \textbf{glpi/glpi} : Super-Administrateur
    \item \textbf{tech/tech} : Technicien
    \item \textbf{normal/normal} : Utilisateur standard
    \item \textbf{post-only/postonly} : Post uniquement
\end{itemize}

\subsection{Création utilisateurs}
Navigation : \textbf{Administration > Utilisateurs > +}

Exemple utilisateur admin :
\begin{itemize}
    \item Nom d'utilisateur : admin\_si
    \item Nom : Administrateur
    \item Prénom : Système
    \item Email : admin@entreprise.local
    \item Profil : Super-Admin
    \item Statut : Actif
\end{itemize}

% ========================================
% INVENTAIRE
% ========================================
\section{Gestion de l'inventaire}

\subsection{Agent GLPI (Windows)}
\begin{lstlisting}[language=bash]
# Téléchargement et installation
Invoke-WebRequest -Uri "https://github.com/glpi-project/glpi-agent/releases/download/1.7.3/GLPI-Agent-1.7.3-x64.msi" -OutFile "GLPI-Agent.msi"

# Installation avec configuration
msiexec /i GLPI-Agent.msi /quiet SERVER="http://10.100.133.85/glpi/front/inventory.php" TAG="Production"
\end{lstlisting}

\subsection{Agent GLPI (Linux)}
\begin{lstlisting}[language=bash]
# Installation
wget -O glpi-agent.deb https://github.com/glpi-project/glpi-agent/releases/download/1.7.3/glpi-agent_1.7.3-1_amd64.deb
sudo dpkg -i glpi-agent.deb

# Configuration
sudo nano /etc/glpi-agent/agent.cfg
\end{lstlisting}

Configuration agent.cfg :
\begin{lstlisting}
server = http://10.100.133.85/glpi/front/inventory.php
tag = Production
\end{lstlisting}

\begin{lstlisting}[language=bash]
sudo systemctl restart glpi-agent
sudo systemctl enable glpi-agent
\end{lstlisting}

\subsection{Ajout manuel d'équipements}
Navigation : \textbf{Parc > Ordinateurs > +}

Exemple fiche poste :
\begin{itemize}
    \item Nom : PC-DIRECTION-01
    \item Série : ABC123456789
    \item Fabricant : Dell
    \item Modèle : OptiPlex 7090
    \item Utilisateur : Jean Martin
    \item Lieu : Bureau direction
    \item IP : 10.100.133.10
\end{itemize}

% ========================================
% TESTS ET MAINTENANCE
% ========================================
\section{Tests et maintenance}

\subsection{Tests de validation}
\begin{lstlisting}[language=bash]
# Test base de données
sudo mysql -u glpiuser -p -e "USE glpidb; SHOW TABLES;" | wc -l

# Test accès web
curl -I http://10.100.133.85/glpi/

# Test services
sudo systemctl status httpd mariadb

# Test permissions
sudo -u http ls -la /srv/http/glpi/files/
\end{lstlisting}

\subsection{Maintenance courante}
\begin{lstlisting}[language=bash]
# Redémarrage services
sudo systemctl restart httpd mariadb

# Mise à jour système
sudo pacman -Syu

# Vérification logs
tail -f /srv/http/glpi/files/_log/glpi.log
sudo journalctl -u httpd -f

# Réparation permissions
sudo chown -R http:http /srv/http/glpi/
sudo chmod -R 755 /srv/http/glpi/
sudo chmod -R 775 /srv/http/glpi/files/ /srv/http/glpi/config/
\end{lstlisting}

\subsection{Informations importantes}
\begin{itemize}
    \item \textbf{URL GLPI :} http://10.100.133.85/glpi/
    \item \textbf{URL agents :} http://10.100.133.85/glpi/front/inventory.php
    \item \textbf{Répertoire :} /srv/http/glpi/
    \item \textbf{Utilisateur Apache :} http
    \item \textbf{Base de données :} glpidb sur localhost:3306
    \item \textbf{Logs :} /srv/http/glpi/files/\_log/glpi.log
\end{itemize}

\section{Conclusion}

Installation GLPI 10.0.20 sur Arch Linux avec Apache httpd opérationnelle. Configuration spécifique adaptée à l'environnement avec accès HTTP direct via 10.100.133.85. Système prêt pour la gestion du patrimoine informatique avec inventaire automatique et gestion des utilisateurs.

\end{document}