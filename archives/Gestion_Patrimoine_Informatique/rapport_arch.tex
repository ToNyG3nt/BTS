\documentclass[12pt,a4paper]{article}
\usepackage[utf8]{inputenc}
\usepackage[french]{babel}
\usepackage[T1]{fontenc}
\usepackage{geometry}
\usepackage{graphicx}
\usepackage{fancyhdr}
\usepackage{listings}
\usepackage{xcolor}
\usepackage{hyperref}

% Configuration de la page
\geometry{left=2.5cm, right=2.5cm, top=3cm, bottom=2.5cm}
\setlength{\headheight}{15pt}

% Configuration des en-têtes et pieds de page
\pagestyle{fancy}
\fancyhf{}
\fancyhead[L]{Gestion du Patrimoine Informatique}
\fancyhead[R]{BTS SIO}
\fancyfoot[C]{\thepage}

% Configuration pour le code
\lstdefinestyle{mystyle}{
    backgroundcolor=\color{gray!10},
    commentstyle=\color{green!60!black},
    keywordstyle=\color{blue},
    numberstyle=\tiny\color{gray},
    stringstyle=\color{red},
    basicstyle=\ttfamily\footnotesize,
    breakatwhitespace=false,
    breaklines=true,
    captionpos=b,
    keepspaces=true,
    numbers=left,
    numbersep=5pt,
    showspaces=false,
    showstringspaces=false,
    showtabs=false,
    tabsize=2,
    frame=single,
    rulecolor=\color{black}
}
\lstset{style=mystyle}

% Style pour les encadrés
\newcommand{\infobox}[2]{\subsection{INFO - #1}\textit{#2}}
\newcommand{\warningbox}[2]{\subsection{ATTENTION - #1}\textit{#2}}
\newcommand{\successbox}[2]{\subsection{SUCCES - #1}\textit{#2}}

\title{
    \LARGE{\textbf{Gestion du Patrimoine Informatique}} \\
    \Large{Installation GLPI sur Arch Linux} \\
    \Large{GLPI 10.0.20 avec Apache HTTP}
}

\author{
    \textbf{[Votre Nom]} \\
    BTS SIO - Services Informatiques aux Organisations
}

\date{\today}

\begin{document}

\maketitle
\thispagestyle{empty}

\newpage
\tableofcontents
\newpage

% ========================================
% INTRODUCTION
% ========================================
\section{Introduction}

\subsection{Contexte du projet}
Dans le cadre de la formation BTS SIO, ce rapport présente la mise en place d'une solution de gestion du patrimoine informatique basée sur GLPI (Gestionnaire Libre de Parc Informatique) version 10.0.20. Cette installation est réalisée sur un serveur Arch Linux avec Apache HTTP, accessible via l'adresse IP 10.100.133.85.

\subsection{Configuration spécifique}
\begin{itemize}
    \item \textbf{Système d'exploitation} : Arch Linux
    \item \textbf{Serveur web} : Apache (httpd)
    \item \textbf{Version GLPI} : 10.0.20
    \item \textbf{Base de données} : MariaDB
    \item \textbf{URL d'accès} : http://10.100.133.85/glpi/index.php?noAUTO=1
    \item \textbf{Protocole} : HTTP (pas de HTTPS)
\end{itemize}

% ========================================
% INSTALLATION DE GLPI - PROCÉDURE COMPLÈTE
% ========================================
\section{Installation de GLPI - Procédure Complète}

\subsection{Étape 1 : Installation de LAMP avec Base de Données MariaDB}

\subsubsection{Installation d'Apache, MariaDB et PHP sur Arch Linux}
\begin{lstlisting}[language=bash, caption=Installation complète de LAMP sur Arch Linux]
# Mise à jour du système
sudo pacman -Syu

# Installation d'Apache
sudo pacman -S apache

# Installation de MariaDB
sudo pacman -S mariadb

# Installation de PHP et toutes les extensions nécessaires
sudo pacman -S php php-apache php-gd php-intl php-zip php-curl php-mbstring php-xml php-mysql php-ldap php-imap php-snmp

# Démarrage et activation des services
sudo systemctl start httpd
sudo systemctl enable httpd
sudo systemctl start mariadb
sudo systemctl enable mariadb
\end{lstlisting}

\subsubsection{Sécurisation et création de la base de données MariaDB}
\begin{lstlisting}[language=bash, caption=Configuration MariaDB pour GLPI]
# Initialisation de MariaDB (première fois)
sudo mysql_install_db --user=mysql --basedir=/usr --datadir=/var/lib/mysql

# Démarrage du service
sudo systemctl start mariadb

# Sécurisation de MariaDB
sudo mysql_secure_installation
# Répondre : Y, nouveau_mot_de_passe, Y, Y, Y, Y

# Connexion à MariaDB et création de la base GLPI
sudo mysql -u root -p
\end{lstlisting}

\begin{lstlisting}[caption=Commandes SQL pour GLPI]
-- Création de la base de données GLPI
CREATE DATABASE glpidb CHARACTER SET utf8mb4 COLLATE utf8mb4_unicode_ci;

-- Création de l'utilisateur GLPI
CREATE USER 'glpiuser'@'localhost' IDENTIFIED BY 'MotDePasseSecurise123!';

-- Attribution des droits complets sur la base GLPI
GRANT ALL PRIVILEGES ON glpidb.* TO 'glpiuser'@'localhost';

-- Application des modifications
FLUSH PRIVILEGES;

-- Vérification de la création
SHOW DATABASES;
SELECT User, Host FROM mysql.user WHERE User = 'glpiuser';

-- Sortie de MariaDB
EXIT;
\end{lstlisting}

\subsection{Étape 2 : Téléchargement de GLPI Version 10.0.20 depuis GitHub}

\subsubsection{Récupération des sources officielles}
\begin{lstlisting}[language=bash, caption=Téléchargement GLPI 10.0.20]
# Déplacement vers le répertoire temporaire
cd /tmp

# Téléchargement de la version exacte 10.0.20 depuis GitHub officiel
wget https://github.com/glpi-project/glpi/releases/download/10.0.20/glpi-10.0.20.tgz

# Vérification du téléchargement
ls -la glpi-10.0.20.tgz

# Extraction de l'archive
tar -xzf glpi-10.0.20.tgz

# Vérification de l'extraction
ls -la glpi/

# Déplacement vers le répertoire web d'Apache (Arch Linux)
sudo mv glpi /srv/http/

# Vérification de l'installation
ls -la /srv/http/glpi/
\end{lstlisting}

\subsection{Étape 3 : Gestion des Droits Utilisateur pour Apache}

\subsubsection{Configuration des permissions}
\begin{lstlisting}[language=bash, caption=Attribution des droits Apache sur Arch]
# Attribution de la propriété à l'utilisateur Apache (http sur Arch)
sudo chown -R http:http /srv/http/glpi/

# Configuration des permissions de base
sudo chmod -R 755 /srv/http/glpi/

# Permissions spéciales pour les répertoires sensibles
sudo chmod -R 775 /srv/http/glpi/files/
sudo chmod -R 775 /srv/http/glpi/config/

# Vérification des permissions
ls -la /srv/http/glpi/
ls -la /srv/http/glpi/files/
ls -la /srv/http/glpi/config/

# Test de l'écriture par Apache
sudo -u http touch /srv/http/glpi/files/test_write.txt
ls -la /srv/http/glpi/files/test_write.txt
sudo rm /srv/http/glpi/files/test_write.txt
\end{lstlisting}

\subsection{Étape 4 : Activation des Extensions PHP Nécessaires}

\subsubsection{Configuration d'Apache et PHP sur Arch Linux}
\begin{lstlisting}[language=bash, caption=Configuration Apache et PHP sur Arch]
# Édition du fichier de configuration Apache principal
sudo nano /etc/httpd/conf/httpd.conf

# Modifications à apporter dans httpd.conf :
# 1. Décommenter : LoadModule rewrite_module modules/mod_rewrite.so
# 2. Ajouter à la fin du fichier : LoadModule php_module modules/libphp.so
# 3. Ajouter : AddHandler php-script .php
# 4. Ajouter : Include conf/extra/php_module.conf

# Créer le fichier de configuration PHP pour Apache
sudo nano /etc/httpd/conf/extra/php_module.conf
\end{lstlisting}

\textbf{Contenu du fichier php\_module.conf :}
\begin{lstlisting}[caption=Configuration PHP pour Apache]
<IfModule php_module>
    DirectoryIndex index.php index.html
    <FilesMatch "\.php$">
        SetHandler application/x-httpd-php
    </FilesMatch>
</IfModule>
\end{lstlisting}

\begin{lstlisting}[language=bash, caption=Test et redémarrage des services]
# Vérification des extensions PHP installées
php -m | grep -E "mysql|mbstring|curl|gd|xml|zip|intl|ldap|imap|snmp"

# Test de la configuration Apache
sudo httpd -t

# Redémarrage d'Apache pour appliquer les modifications
sudo systemctl restart httpd

# Vérification du statut des services
sudo systemctl status httpd mariadb

# Test de la configuration PHP
php -v
php -m | wc -l
\end{lstlisting}

\subsection{Étape 5 : Première Installation via Navigateur}

\subsubsection{Processus d'installation web}
\successbox{Accès à l'interface d'installation}{
URL d'accès : http://10.100.133.85/glpi/index.php?noAUTO=1
}

\textbf{Étapes de l'assistant d'installation :}
\begin{enumerate}
    \item \textbf{Sélection de la langue}
    \begin{itemize}
        \item Ouvrir un navigateur et accéder à l'URL
        \item Choisir "Français"
        \item Cliquer sur "OK"
    \end{itemize}
    
    \item \textbf{Acceptation de la licence}
    \begin{itemize}
        \item Lire et accepter la licence GPL v3+
        \item Cliquer sur "Continuer"
    \end{itemize}
    
    \item \textbf{Vérification des prérequis}
    \begin{itemize}
        \item Vérifier que tous les prérequis sont en vert
        \item Extensions PHP requises : mysql, mbstring, curl, gd, xml, zip, intl
        \item Si des éléments sont en rouge, les corriger avant de continuer
        \item Cliquer sur "Continuer"
    \end{itemize}
    
    \item \textbf{Configuration de la base de données}
    \begin{itemize}
        \item Serveur de BDD : localhost
        \item Utilisateur : glpiuser
        \item Mot de passe : MotDePasseSecurise123!
        \item Base de données : glpidb
        \item Cliquer sur "Continuer"
    \end{itemize}
    
    \item \textbf{Initialisation de la base de données}
    \begin{itemize}
        \item Attendre la création des tables (peut prendre plusieurs minutes)
        \item Vérifier qu'aucune erreur n'apparaît
        \item Cliquer sur "Continuer"
    \end{itemize}
    
    \item \textbf{Fin de l'installation}
    \begin{itemize}
        \item Noter les comptes créés par défaut
        \item Cliquer sur "Utiliser GLPI"
    \end{itemize}
\end{enumerate}

\warningbox{Comptes par défaut créés}{
\begin{itemize}
    \item glpi/glpi : Super-Administrateur (tous droits)
    \item tech/tech : Technicien (gestion technique)
    \item normal/normal : Utilisateur standard
    \item post-only/postonly : Utilisateur post-uniquement
\end{itemize}
}

\subsection{Étape 6 : Connexion Super Admin et Création d'Utilisateurs}

\subsubsection{Première connexion avec le compte Super Admin}
\begin{enumerate}
    \item \textbf{Connexion initiale}
    \begin{itemize}
        \item URL : http://10.100.133.85/glpi/
        \item Utilisateur : glpi
        \item Mot de passe : glpi
        \item Cliquer sur "Se connecter"
    \end{itemize}
    
    \item \textbf{Changement obligatoire du mot de passe}
    \begin{itemize}
        \item Aller dans "Mon profil" (icône utilisateur en haut à droite)
        \item Onglet "Préférences" > "Modifier le mot de passe"
        \item Nouveau mot de passe sécurisé
        \item Confirmer le changement
    \end{itemize}
    
    \item \textbf{Suppression du répertoire d'installation}
    \begin{itemize}
        \item Message d'alerte affiché par GLPI
        \item Exécuter dans le terminal : sudo rm -rf /srv/http/glpi/install/
        \item Actualiser la page web
        \item L'alerte doit disparaître
    \end{itemize}
\end{enumerate}

\subsubsection{Création des nouveaux utilisateurs}
\textbf{Navigation : Administration > Utilisateurs > + (Ajouter)}

\begin{lstlisting}[caption=Exemple de création d'utilisateur administrateur]
Nom d'utilisateur : admin_si
Nom (famille) : Administrateur
Prénom : Système Info
Email : admin@entreprise.local
Mot de passe : [Générer un mot de passe sécurisé]
Confirmer le mot de passe : [Répéter le mot de passe]
Profil par défaut : Super-Admin
Entité par défaut : Entité racine
Statut : Actif
\end{lstlisting}

\subsection{Étape 7 : Création des Utilisateurs et Attribution des Droits}

\subsubsection{Définition des profils par poste}
\begin{center}
\begin{tabular}{|l|l|l|}
\hline
\textbf{Poste} & \textbf{Profil GLPI} & \textbf{Droits attribués} \\
\hline
Directeur IT & Super-Admin & Tous droits, configuration système \\
Responsable IT & Admin & Gestion complète, sans config système \\
Technicien Senior & Technicien & Gestion tickets, inventaire, rapports \\
Technicien & Technicien & Gestion tickets, consultation inventaire \\
Support Helpdesk & Hotliner & Création/modification tickets uniquement \\
Utilisateur final & Self-Service & Création tickets, consultation matériel \\
Invité/Consultant & Observer & Lecture seule \\
\hline
\end{tabular}
\end{center}

\subsubsection{Exemple de création utilisateurs par service}
\begin{lstlisting}[caption=Utilisateurs Service Informatique]
# Directeur IT
Utilisateur : directeur.it
Nom : Martin
Prénom : Jean
Profil : Super-Admin
Entité : Entité racine
Groupes : Direction, Service IT

# Responsable technique
Utilisateur : resp.technique
Nom : Durand
Prénom : Marie
Profil : Admin  
Entité : Service IT
Groupes : Service IT, Responsables

# Technicien senior
Utilisateur : tech.senior1
Nom : Bernard
Prénom : Paul
Profil : Technicien
Entité : Service IT
Groupes : Service IT, Techniciens

# Technicien
Utilisateur : tech1
Nom : Petit
Prénom : Sophie
Profil : Technicien (droits limités)
Entité : Service IT
Groupes : Service IT, Support
\end{lstlisting}

\subsection{Étape 8 : Création des Postes de Travail et Attribution (Inventaire)}

\subsubsection{Configuration de l'inventaire automatique}
\textbf{Installation de l'agent GLPI sur les postes clients :}

\begin{lstlisting}[language=bash, caption=Installation agent GLPI - Windows]
# Téléchargement de l'agent GLPI
# URL : https://github.com/glpi-project/glpi-agent/releases/latest

# Installation silencieuse via PowerShell (Administrateur)
Invoke-WebRequest -Uri "https://github.com/glpi-project/glpi-agent/releases/download/1.7.3/GLPI-Agent-1.7.3-x64.msi" -OutFile "GLPI-Agent.msi"

# Installation avec configuration automatique pour votre serveur
msiexec /i GLPI-Agent.msi /quiet SERVER="http://10.100.133.85/glpi/front/inventory.php" TAG="Production"

# Vérification de l'installation
sc query "GLPI-Agent"
\end{lstlisting}

\begin{lstlisting}[language=bash, caption=Installation agent GLPI - Linux]
# Installation sur Ubuntu/Debian
wget -O glpi-agent.deb https://github.com/glpi-project/glpi-agent/releases/download/1.7.3/glpi-agent_1.7.3-1_amd64.deb
sudo dpkg -i glpi-agent.deb
sudo apt-get install -f

# Configuration de l'agent
sudo nano /etc/glpi-agent/agent.cfg

# Contenu du fichier de configuration pour votre serveur :
server = http://10.100.133.85/glpi/front/inventory.php
tag = Production
logger = stderr
logfile-maxsize = 10485760

# Redémarrage de l'agent
sudo systemctl restart glpi-agent
sudo systemctl enable glpi-agent

# Test de l'agent
glpi-agent --test --debug
\end{lstlisting}

\subsubsection{Création manuelle des postes de travail}
\textbf{Navigation : Parc > Ordinateurs > + (Ajouter)}

\begin{lstlisting}[caption=Exemple de fiche poste de travail]
Nom : PC-DIRECTION-01
Numéro de série : ABC123456789
Numéro d'inventaire : INV-2024-001
Fabricant : Dell
Modèle : OptiPlex 7090
Type : Ordinateur de bureau
Statut : En fonction
Lieu : Bureau direction
Utilisateur : Jean Martin (Directeur IT)
Groupe : Direction
Domaine : ENTREPRISE.LOCAL
Système d'exploitation : Windows 11 Pro 22H2
Processeur : Intel Core i7-11700 @ 2.50GHz
Mémoire : 16384 Mo DDR4
Disque dur : SSD NVMe 512 Go
Carte réseau : Intel I219-LM
Adresse IP : 10.100.133.10
\end{lstlisting}

\subsubsection{Attribution des équipements aux utilisateurs}
\textbf{Processus d'attribution dans GLPI :}

\begin{enumerate}
    \item \textbf{Accéder à la fiche de l'ordinateur}
    \begin{itemize}
        \item Parc > Ordinateurs > Cliquer sur le nom de l'ordinateur
    \end{itemize}
    
    \item \textbf{Onglet "Gestion"}
    \begin{itemize}
        \item Cliquer sur l'onglet "Gestion" dans la fiche
    \end{itemize}
    
    \item \textbf{Remplir les informations d'attribution :}
    \begin{itemize}
        \item Utilisateur : Sélectionner dans la liste déroulante
        \item Groupe : Groupe d'appartenance de l'utilisateur
        \item Lieu : Localisation physique précise
        \item Statut : "Attribué" ou "En fonction"
        \item Date d'attribution : Date du jour
    \end{itemize}
    
    \item \textbf{Validation}
    \begin{itemize}
        \item Cliquer sur "Actualiser" pour enregistrer
        \item Vérifier que l'attribution apparaît dans l'historique
    \end{itemize}
\end{enumerate}

\successbox{Vérification de l'installation complète}{
Points de contrôle après installation sur Arch Linux :
\begin{itemize}
    \item LAMP fonctionnel (Apache httpd + MariaDB + PHP)
    \item GLPI 10.0.20 accessible via http://10.100.133.85/glpi/
    \item Base de données glpidb opérationnelle
    \item Utilisateurs créés avec les profils appropriés
    \item Interface web réactive et fonctionnelle
    \item Répertoire install/ supprimé pour la sécurité
    \item Agents prêts pour déploiement automatique
\end{itemize}
}

% ========================================
% TESTS ET VALIDATION
% ========================================
\section{Tests et Validation Post-Installation}

\subsection{Tests fonctionnels}
\begin{lstlisting}[language=bash, caption=Tests de validation système sur Arch]
# Test de la base de données
sudo mysql -u glpiuser -p -e "USE glpidb; SHOW TABLES;" | wc -l
# Doit retourner environ 400+ tables

# Test de l'accès web GLPI
curl -I http://10.100.133.85/glpi/
# Doit retourner : HTTP/1.1 200 OK

# Test des services Arch
sudo systemctl status httpd mariadb

# Test des permissions fichiers
sudo -u http ls -la /srv/http/glpi/files/
# Vérifier que http peut lire/écrire

# Test PHP dans Apache
echo "<?php phpinfo(); ?>" | sudo tee /srv/http/test.php
curl http://10.100.133.85/test.php | grep "PHP Version"
sudo rm /srv/http/test.php
\end{lstlisting}

\subsection{Validation des fonctionnalités}
\textbf{Tests à effectuer via l'interface web http://10.100.133.85/glpi/ :}
\begin{enumerate}
    \item \textbf{Connexion des différents utilisateurs créés}
    \item \textbf{Création d'un ticket de test}
    \item \textbf{Ajout manuel d'un équipement dans l'inventaire}
    \item \textbf{Navigation dans les différents menus}
    \item \textbf{Test des droits selon les profils}
    \item \textbf{Vérification des logs système}
\end{enumerate}

\subsection{Dépannage spécifique Arch Linux}

\subsubsection{Problèmes courants et solutions}
\begin{lstlisting}[language=bash, caption=Dépannage Arch Linux]
# Si Apache ne démarre pas
sudo systemctl status httpd
sudo journalctl -u httpd

# Si PHP n'est pas reconnu
sudo nano /etc/httpd/conf/httpd.conf
# Vérifier les lignes LoadModule php_module

# Si MariaDB refuse les connexions
sudo systemctl restart mariadb
sudo mysql -u root -p -e "FLUSH PRIVILEGES;"

# Vérification des logs GLPI
tail -f /srv/http/glpi/files/_log/glpi.log

# Permissions après modification
sudo chown -R http:http /srv/http/glpi/
sudo chmod -R 755 /srv/http/glpi/
sudo chmod -R 775 /srv/http/glpi/files/ /srv/http/glpi/config/
\end{lstlisting}

% ========================================
% CONCLUSION
% ========================================
\section{Conclusion}

La mise en place de GLPI version 10.0.20 sur Arch Linux suivant ces 8 étapes détaillées permet d'obtenir une solution complète et opérationnelle de gestion du patrimoine informatique. L'installation spécifique sur Arch Linux avec Apache httpd garantit :

\begin{itemize}
    \item Une base technique solide avec LAMP correctement configuré sur Arch
    \item Configuration Apache native (httpd) avec PHP intégré
    \item Accès HTTP direct via l'IP 10.100.133.85 sans complexité de certificats
    \item Un système d'inventaire automatique prêt pour le déploiement
    \item Une organisation claire des utilisateurs et de leurs droits
    \item Un processus d'attribution des équipements structuré
    \item Une solution adaptée à l'environnement technique spécifique
\end{itemize}

Cette configuration répond parfaitement aux besoins d'une infrastructure moderne tout en conservant la simplicité d'installation et de maintenance propre à Arch Linux.

% ========================================
% ANNEXES
% ========================================
\newpage
\section{Annexes}

\subsection{Annexe A : Commandes de maintenance Arch Linux}

\begin{lstlisting}[language=bash, caption=Maintenance GLPI sur Arch]
# Vérification des services
sudo systemctl status httpd mariadb

# Redémarrage des services
sudo systemctl restart httpd mariadb

# Mise à jour du système Arch
sudo pacman -Syu

# Vérification des logs
sudo journalctl -u httpd -f
sudo journalctl -u mariadb -f
tail -f /srv/http/glpi/files/_log/glpi.log

# Maintenance permissions
sudo chown -R http:http /srv/http/glpi/
sudo chmod -R 755 /srv/http/glpi/
\end{lstlisting}

\subsection{Annexe B : Configuration réseau}

\begin{lstlisting}[caption=Informations réseau de l'installation]
Adresse IP serveur : 10.100.133.85
URL d'accès GLPI : http://10.100.133.85/glpi/index.php?noAUTO=1
URL agents : http://10.100.133.85/glpi/front/inventory.php
Répertoire web : /srv/http/glpi/
Utilisateur Apache : http
Groupe Apache : http
Base de données : glpidb sur localhost:3306
\end{lstlisting}

\subsection{Annexe C : Références}

\begin{itemize}
    \item Documentation GLPI : \url{https://glpi-project.org/documentation/}
    \item GitHub GLPI : \url{https://github.com/glpi-project/glpi}
    \item Wiki Arch Linux Apache : \url{https://wiki.archlinux.org/title/Apache_HTTP_Server}
    \item Wiki Arch Linux MariaDB : \url{https://wiki.archlinux.org/title/MariaDB}
    \item GLPI Agents : \url{https://github.com/glpi-project/glpi-agent}
\end{itemize}

\end{document}