\documentclass[12pt,a4paper]{article}
\usepackage[utf8]{inputenc}
\usepackage[french]{babel}
\usepackage[T1]{fontenc}
\usepackage{geometry}
\usepackage{graphicx}
\usepackage{fancyhdr}
\usepackage{listings}
\usepackage{xcolor}
\usepackage{hyperref}

% Configuration de la page
\geometry{left=2.5cm, right=2.5cm, top=3cm, bottom=2.5cm}
\setlength{\headheight}{15pt}

% Configuration des en-têtes et pieds de page
\pagestyle{fancy}
\fancyhf{}
\fancyhead[L]{Gestion du Patrimoine Informatique}
\fancyhead[R]{BTS SIO}
\fancyfoot[C]{\thepage}

% Configuration pour le code
\lstdefinestyle{mystyle}{
    backgroundcolor=\color{gray!10},
    commentstyle=\color{green!60!black},
    keywordstyle=\color{blue},
    numberstyle=\tiny\color{gray},
    stringstyle=\color{red},
    basicstyle=\ttfamily\footnotesize,
    breakatwhitespace=false,
    breaklines=true,
    captionpos=b,
    keepspaces=true,
    numbers=left,
    numbersep=5pt,
    showspaces=false,
    showstringspaces=false,
    showtabs=false,
    tabsize=2,
    frame=single,
    rulecolor=\color{black}
}
\lstset{style=mystyle}

% Style pour les encadrés
\newcommand{\infobox}[2]{\subsection{#1}\textit{#2}}
\newcommand{\warningbox}[2]{\subsection{ATTENTION - #1}\textit{#2}}
\newcommand{\successbox}[2]{\subsection{SUCCES - #1}\textit{#2}}

\title{
    \LARGE{\textbf{Gestion du Patrimoine Informatique}} \\
    \Large{Installation et Hébergement de GLPI} \\
    \Large{Création et Gestion d'un Parc Informatique}
}

\author{
    \textbf{[Votre Nom]} \\
    BTS SIO - Services Informatiques aux Organisations
}

\date{\today}

\begin{document}

\maketitle
\thispagestyle{empty}

\newpage
\tableofcontents
\newpage

% ========================================
% INTRODUCTION
% ========================================
\section{Introduction}

\subsection{Contexte du projet}
Dans le cadre de la formation BTS SIO, ce rapport présente la mise en place d'une solution de gestion du patrimoine informatique basée sur GLPI (Gestionnaire Libre de Parc Informatique). Cette solution permet de centraliser et d'automatiser la gestion des équipements informatiques d'une organisation.

\subsection{Objectifs}
\begin{itemize}
    \item Installer et configurer GLPI sur un serveur dédié
    \item Mettre en place l'hébergement et la sécurisation de l'application
    \item Créer un inventaire complet du parc informatique
    \item Implémenter les processus de gestion des équipements
    \item Former les utilisateurs à l'utilisation de l'outil
\end{itemize}

\subsection{Périmètre}
Ce projet couvre :
\begin{itemize}
    \item L'installation technique de GLPI
    \item La configuration de l'environnement d'hébergement
    \item La création de la base de données du parc
    \item La mise en place des procédures de gestion
\end{itemize}

% ========================================
% ANALYSE ET CONCEPTION
% ========================================
\section{Analyse et Conception}

\subsection{Analyse des besoins}

\subsubsection{Besoins fonctionnels}
\infobox{Fonctionnalités principales}{
\begin{itemize}
    \item \textbf{Inventaire automatisé} : Découverte et recensement automatique des équipements
    \item \textbf{Gestion des tickets} : Système de helpdesk intégré
    \item \textbf{Suivi des contrats} : Gestion des garanties et contrats de maintenance
    \item \textbf{Gestion des utilisateurs} : Attribution et suivi des équipements par utilisateur
    \item \textbf{Rapports et statistiques} : Tableaux de bord et reporting
\end{itemize}
}

\subsubsection{Besoins techniques}
\begin{itemize}
    \item Serveur web (Apache/Nginx)
    \item Base de données MySQL/MariaDB
    \item PHP 7.4 ou supérieur
    \item Agent GLPI pour l'inventaire automatique
    \item Connexion réseau sécurisée (HTTPS)
\end{itemize}

\subsection{Architecture technique}

\subsubsection{Infrastructure}
\begin{center}
\begin{tabular}{|l|l|}
\hline
\textbf{Composant} & \textbf{Spécification} \\
\hline
Serveur & [Indiquer les caractéristiques : CPU, RAM, Stockage] \\
OS & [Ubuntu Server 20.04 LTS / CentOS 8 / autre] \\
Serveur Web & Apache 2.4 / Nginx \\
Base de données & MySQL 8.0 / MariaDB 10.5 \\
PHP & Version 8.0+ \\
\hline
\end{tabular}
\end{center}

% ========================================
% INSTALLATION DE GLPI - PROCÉDURE COMPLÈTE
% ========================================
\section{Installation de GLPI - Procédure Complète}

\subsection{Étape 1 : Installation de LAMP avec Base de Données MariaDB}

\subsubsection{Installation d'Apache, MariaDB et PHP}
\begin{lstlisting}[language=bash, caption=Installation complète de LAMP]
# Mise à jour du système
sudo apt update && sudo apt upgrade -y

# Installation d'Apache
sudo apt install apache2 -y

# Installation de MariaDB
sudo apt install mariadb-server -y

# Installation de PHP et toutes les extensions nécessaires
sudo apt install php libapache2-mod-php php-mysql php-mbstring php-curl php-gd php-xml php-zip php-intl php-ldap php-imap php-snmp php-cas php-apcu -y

# Démarrage et activation des services
sudo systemctl start apache2
sudo systemctl enable apache2
sudo systemctl start mariadb
sudo systemctl enable mariadb
\end{lstlisting}

\subsubsection{Sécurisation et création de la base de données MariaDB}
\begin{lstlisting}[language=bash, caption=Configuration MariaDB pour GLPI]
# Sécurisation de MariaDB
sudo mysql_secure_installation
# Répondre : Y, nouveau_mot_de_passe, Y, Y, Y, Y

# Connexion à MariaDB et création de la base GLPI
sudo mysql -u root -p
\end{lstlisting}

\begin{lstlisting}[caption=Commandes SQL pour GLPI]
-- Création de la base de données GLPI
CREATE DATABASE glpidb CHARACTER SET utf8mb4 COLLATE utf8mb4_unicode_ci;

-- Création de l'utilisateur GLPI
CREATE USER 'glpiuser'@'localhost' IDENTIFIED BY 'MotDePasseSecurise123!';

-- Attribution des droits complets sur la base GLPI
GRANT ALL PRIVILEGES ON glpidb.* TO 'glpiuser'@'localhost';

-- Application des modifications
FLUSH PRIVILEGES;

-- Vérification de la création
SHOW DATABASES;
SELECT User, Host FROM mysql.user WHERE User = 'glpiuser';

-- Sortie de MariaDB
EXIT;
\end{lstlisting}

\subsection{Étape 2 : Téléchargement de GLPI Version 10.0.20 depuis GitHub}

\subsubsection{Récupération des sources officielles}
\begin{lstlisting}[language=bash, caption=Téléchargement GLPI 10.0.20]
# Déplacement vers le répertoire temporaire
cd /tmp

# Téléchargement de la version exacte 10.0.20 depuis GitHub officiel
wget https://github.com/glpi-project/glpi/releases/download/10.0.20/glpi-10.0.20.tgz

# Vérification du téléchargement
ls -la glpi-10.0.20.tgz

# Extraction de l'archive
tar -xzf glpi-10.0.20.tgz

# Vérification de l'extraction
ls -la glpi/

# Déplacement vers le répertoire web d'Apache
sudo mv glpi /var/www/html/

# Vérification de l'installation
ls -la /var/www/html/glpi/
\end{lstlisting}

\subsection{Étape 3 : Gestion des Droits Utilisateur pour Apache}

\subsubsection{Configuration des permissions}
\begin{lstlisting}[language=bash, caption=Attribution des droits Apache]
# Attribution de la propriété à l'utilisateur Apache (www-data)
sudo chown -R www-data:www-data /var/www/html/glpi/

# Configuration des permissions de base
sudo chmod -R 755 /var/www/html/glpi/

# Permissions spéciales pour les répertoires sensibles
sudo chmod -R 775 /var/www/html/glpi/files/
sudo chmod -R 775 /var/www/html/glpi/config/

# Vérification des permissions
ls -la /var/www/html/glpi/
ls -la /var/www/html/glpi/files/
ls -la /var/www/html/glpi/config/

# Test de l'écriture par Apache
sudo -u www-data touch /var/www/html/glpi/files/test_write.txt
ls -la /var/www/html/glpi/files/test_write.txt
sudo rm /var/www/html/glpi/files/test_write.txt
\end{lstlisting}

\subsection{Étape 4 : Activation des Extensions PHP Nécessaires}

\subsubsection{Vérification et activation des modules PHP}
\begin{lstlisting}[language=bash, caption=Configuration des extensions PHP]
# Vérification des extensions PHP installées
php -m | grep -E "mysql|mbstring|curl|gd|xml|zip|intl|ldap|imap|snmp|cas|apcu"

# Activation des modules Apache PHP si nécessaires
sudo a2enmod php8.1  # ou php8.0 selon votre version
sudo a2enmod rewrite
sudo a2enmod headers

# Redémarrage d'Apache pour appliquer les modifications
sudo systemctl restart apache2

# Création d'un fichier de test PHP
sudo nano /var/www/html/info.php
\end{lstlisting}

\textbf{Contenu du fichier info.php :}
\begin{lstlisting}[caption=Fichier de test PHP]
<?php
phpinfo();
?>
\end{lstlisting}

\begin{lstlisting}[language=bash, caption=Test des extensions PHP]
# Test de la configuration PHP via navigateur
# Accéder à : http://votre-serveur/info.php

# Vérification en ligne de commande
php -v
php -m | wc -l  # Nombre d'extensions chargées

# Suppression du fichier de test (sécurité)
sudo rm /var/www/html/info.php
\end{lstlisting}

\subsection{Étape 5 : Première Installation via Navigateur}

\subsubsection{Processus d'installation web}
\successbox{Accès à l'interface d'installation}{
\textbf{URL d'accès :} \texttt{http://adresse-ip-serveur/glpi} ou \texttt{http://nom-serveur/glpi}
}

\textbf{Étapes de l'assistant d'installation :}
\begin{enumerate}
    \item \textbf{Sélection de la langue}
    \begin{itemize}
        \item Choisir "Français"
        \item Cliquer sur "OK"
    \end{itemize}
    
    \item \textbf{Acceptation de la licence}
    \begin{itemize}
        \item Lire et accepter la licence GPL
        \item Cliquer sur "Continuer"
    \end{itemize}
    
    \item \textbf{Vérification des prérequis}
    \begin{itemize}
        \item Vérifier que tous les prérequis sont en vert
        \item Si des éléments sont en rouge, les corriger avant de continuer
        \item Cliquer sur "Continuer"
    \end{itemize}
    
    \item \textbf{Configuration de la base de données}
    \begin{itemize}
        \item Serveur de BDD : \texttt{localhost}
        \item Utilisateur : \texttt{glpiuser}
        \item Mot de passe : \texttt{MotDePasseSecurise123!}
        \item Base de données : \texttt{glpidb}
        \item Cliquer sur "Continuer"
    \end{itemize}
    
    \item \textbf{Initialisation de la base de données}
    \begin{itemize}
        \item Attendre la création des tables
        \item Cliquer sur "Continuer"
    \end{itemize}
    
    \item \textbf{Fin de l'installation}
    \begin{itemize}
        \item Noter les comptes créés par défaut
        \item Cliquer sur "Utiliser GLPI"
    \end{itemize}
\end{enumerate}

\warningbox{Comptes par défaut créés}{
\begin{itemize}
    \item \textbf{glpi/glpi} : Super-Administrateur (tous droits)
    \item \textbf{tech/tech} : Technicien (gestion technique)
    \item \textbf{normal/normal} : Utilisateur standard
    \item \textbf{post-only/postonly} : Utilisateur post-uniquement
\end{itemize}
}

\subsection{Étape 6 : Connexion Super Admin et Création d'Utilisateurs}

\subsubsection{Première connexion avec le compte Super Admin}
\begin{enumerate}
    \item \textbf{Connexion initiale}
    \begin{itemize}
        \item Utilisateur : \texttt{glpi}
        \item Mot de passe : \texttt{glpi}
        \item Cliquer sur "Se connecter"
    \end{itemize}
    
    \item \textbf{Changement obligatoire du mot de passe}
    \begin{itemize}
        \item Aller dans "Mon profil" > "Modifier le mot de passe"
        \item Nouveau mot de passe sécurisé
        \item Confirmer le changement
    \end{itemize}
    
    \item \textbf{Suppression du répertoire d'installation}
    \begin{itemize}
        \item Message d'alerte affiché par GLPI
        \item Exécuter : \texttt{sudo rm -rf /var/www/html/glpi/install/}
        \item Actualiser la page
    \end{itemize}
\end{enumerate}

\subsubsection{Création des nouveaux utilisateurs}
\textbf{Navigation : Administration > Utilisateurs > Créer un utilisateur}

\begin{lstlisting}[caption=Exemple de création d'utilisateur]
Nom d'utilisateur : admin_si
Nom : Administrateur
Prénom : Système Info
Email : admin@entreprise.com
Mot de passe : [Générer un mot de passe sécurisé]
Profil : Super-Admin
Entité : Entité racine
Statut : Actif
\end{lstlisting}

\subsection{Étape 7 : Création des Utilisateurs et Attribution des Droits}

\subsubsection{Définition des profils par poste}
\begin{center}
\begin{tabular}{|l|l|l|}
\hline
\textbf{Poste} & \textbf{Profil GLPI} & \textbf{Droits attribués} \\
\hline
Directeur IT & Super-Admin & Tous droits, configuration système \\
Chef de projet & Admin & Gestion complète, pas de config système \\
Technicien Senior & Technicien & Gestion tickets, inventaire, rapports \\
Technicien Junior & Technicien & Gestion tickets, consultation inventaire \\
Support N1 & Hotliner & Création/modification tickets uniquement \\
Utilisateur final & Self-Service & Création tickets, consultation matériel \\
Invité/Stagiaire & Observer & Lecture seule \\
\hline
\end{tabular}
\end{center}

\subsection{Étape 8 : Création des Postes de Travail et Attribution (Inventaire)}

\subsubsection{Configuration de l'inventaire automatique}
\textbf{Installation de l'agent GLPI sur les postes clients :}

\begin{lstlisting}[language=bash, caption=Installation agent GLPI - Windows]
# Téléchargement de l'agent GLPI
# URL : https://github.com/glpi-project/glpi-agent/releases

# Installation silencieuse via PowerShell (Administrateur)
Invoke-WebRequest -Uri "https://github.com/glpi-project/glpi-agent/releases/download/1.7.3/GLPI-Agent-1.7.3-x64.msi" -OutFile "GLPI-Agent.msi"

# Installation avec configuration automatique
msiexec /i GLPI-Agent.msi /quiet SERVER="http://votre-serveur/glpi/front/inventory.php" TAG="Production"
\end{lstlisting}

\subsubsection{Création manuelle des postes de travail}
\textbf{Navigation : Parc > Ordinateurs > Créer un ordinateur}

\begin{lstlisting}[caption=Exemple de fiche poste de travail]
Nom : PC-COMPTABLE-01
Numéro de série : ABC123456789
Numéro d'inventaire : INV-2024-001
Modèle : Dell OptiPlex 7090
Type : Ordinateur de bureau
Statut : En fonction
Lieu : Bureau comptabilité
Utilisateur : Jean Dupont (Comptable)
Groupe : Service Comptabilité
Domaine : ENTREPRISE.LOCAL
Système d'exploitation : Windows 11 Pro
Processeur : Intel Core i7-11700
Mémoire : 16 Go DDR4
Disque dur : SSD 512 Go
\end{lstlisting}

\successbox{Vérification de l'installation complète}{
\textbf{Points de contrôle après installation :}
\begin{itemize}
    \item LAMP fonctionnel (Apache + MariaDB + PHP)
    \item GLPI 10.0.20 installé et accessible
    \item Base de données configurée et opérationnelle
    \item Utilisateurs créés avec les bons profils
    \item Agents d'inventaire déployés
    \item Premier inventaire automatique réalisé
    \item Attribution des équipements effectuée
\end{itemize}
}

\subsection{Téléchargement et installation de GLPI}

\subsubsection{Récupération des sources}
\begin{lstlisting}[language=bash, caption=Téléchargement de GLPI]
# Téléchargement de la dernière version
cd /tmp
wget https://github.com/glpi-project/glpi/releases/download/10.0.3/glpi-10.0.3.tgz

# Extraction et déplacement
tar -xzf glpi-10.0.3.tgz
sudo mv glpi /var/www/html/

# Configuration des permissions
sudo chown -R www-data:www-data /var/www/html/glpi
sudo chmod -R 755 /var/www/html/glpi
\end{lstlisting}

\subsubsection{Configuration Apache}
\begin{lstlisting}[language=bash, caption=Configuration du VirtualHost]
# Création du fichier de configuration
sudo nano /etc/apache2/sites-available/glpi.conf

# Contenu du fichier VirtualHost
<VirtualHost *:80>
    ServerName glpi.monentreprise.local
    DocumentRoot /var/www/html/glpi
    
    <Directory /var/www/html/glpi>
        Options -Indexes +FollowSymLinks
        AllowOverride All
        Require all granted
    </Directory>
    
    ErrorLog ${APACHE_LOG_DIR}/glpi_error.log
    CustomLog ${APACHE_LOG_DIR}/glpi_access.log combined
</VirtualHost>
\end{lstlisting}

\begin{lstlisting}[language=bash, caption=Activation du site]
# Activation du site et des modules nécessaires
sudo a2ensite glpi.conf
sudo a2enmod rewrite
sudo systemctl restart apache2
\end{lstlisting}

\subsection{Configuration initiale}

\subsubsection{Installation via interface web}
\begin{enumerate}
    \item Accéder à l'URL : \texttt{http://glpi.monentreprise.local}
    \item Suivre l'assistant d'installation :
    \begin{itemize}
        \item Sélectionner la langue (Français)
        \item Accepter la licence GPL
        \item Vérifier les prérequis système
        \item Configurer la connexion à la base de données
        \item Initialiser la base de données
    \end{itemize}
    \item Comptes par défaut créés :
    \begin{itemize}
        \item \textbf{glpi/glpi} (Super-Admin)
        \item \textbf{tech/tech} (Technicien)
        \item \textbf{normal/normal} (Utilisateur normal)
        \item \textbf{post-only/postonly} (Post-uniquement)
    \end{itemize}
\end{enumerate}

\successbox{Sécurisation post-installation}{
Après l'installation, supprimer le répertoire \texttt{install/} :
\begin{lstlisting}[language=bash]
sudo rm -rf /var/www/html/glpi/install/
\end{lstlisting}
}

% ========================================
% HÉBERGEMENT ET SÉCURISATION
% ========================================
\section{Hébergement et Sécurisation}

\subsection{Configuration HTTPS}

\subsubsection{Installation du certificat SSL}
\begin{lstlisting}[language=bash, caption=Configuration Let's Encrypt]
# Installation de Certbot
sudo apt install certbot python3-certbot-apache -y

# Obtention du certificat SSL
sudo certbot --apache -d glpi.monentreprise.local

# Configuration du renouvellement automatique
sudo crontab -e
# Ajouter la ligne :
0 12 * * * /usr/bin/certbot renew --quiet
\end{lstlisting}

\subsection{Sécurisation de l'application}

\subsubsection{Configuration PHP}
\begin{lstlisting}[caption=Modifications dans php.ini]
# Édition du fichier php.ini
sudo nano /etc/php/8.0/apache2/php.ini

# Paramètres de sécurité recommandés
expose_php = Off
allow_url_fopen = Off
allow_url_include = Off
file_uploads = On
upload_max_filesize = 20M
post_max_size = 25M
max_execution_time = 300
memory_limit = 256M
session.cookie_httponly = 1
session.cookie_secure = 1
\end{lstlisting}

\subsection{Sauvegarde et monitoring}

\subsubsection{Script de sauvegarde}
\begin{lstlisting}[language=bash, caption=Script de sauvegarde automatique]
#!/bin/bash
# Script de sauvegarde GLPI

DATE=$(date +%Y%m%d_%H%M%S)
BACKUP_DIR="/backup/glpi"
DB_NAME="glpidb"
DB_USER="glpiuser"
DB_PASS="motdepasse_securise"

# Création du répertoire de sauvegarde
mkdir -p $BACKUP_DIR

# Sauvegarde de la base de données
mysqldump -u $DB_USER -p$DB_PASS $DB_NAME > $BACKUP_DIR/glpi_db_$DATE.sql

# Sauvegarde des fichiers
tar -czf $BACKUP_DIR/glpi_files_$DATE.tar.gz /var/www/html/glpi

# Nettoyage des anciennes sauvegardes (garde 30 jours)
find $BACKUP_DIR -name "*.sql" -mtime +30 -delete
find $BACKUP_DIR -name "*.tar.gz" -mtime +30 -delete
\end{lstlisting}

% ========================================
% CRÉATION DU PARC INFORMATIQUE
% ========================================
\section{Création et Gestion du Parc Informatique}

\subsection{Configuration des entités}

\subsubsection{Structure organisationnelle}
\infobox{Organisation du parc}{
Configuration de l'arborescence des entités dans GLPI :
\begin{itemize}
    \item \textbf{Entité racine} : [Nom de l'organisation]
    \item \textbf{Sites} : Différents sites géographiques
    \item \textbf{Départements} : Services par site
    \item \textbf{Sous-entités} : Équipes spécialisées
\end{itemize}
}

\subsection{Inventaire des équipements}

\subsubsection{Installation de l'agent GLPI}
\begin{lstlisting}[language=bash, caption=Installation sur les postes clients Windows]
# Téléchargement de l'agent GLPI
# Via PowerShell administrateur
Invoke-WebRequest -Uri "https://github.com/glpi-project/glpi-agent/releases/download/1.4/GLPI-Agent-1.4-x64.msi" -OutFile "GLPI-Agent.msi"

# Installation silencieuse
msiexec /i GLPI-Agent.msi /quiet SERVER="https://glpi.monentreprise.local/front/inventory.php" TAG="Production"
\end{lstlisting}

\subsubsection{Configuration pour Linux}
\begin{lstlisting}[language=bash, caption=Installation sur Ubuntu/Debian]
# Installation via le dépôt officiel
wget -O - https://forge.glpi-project.org/attachments/download/2020/glpi-agent.key | sudo apt-key add -
echo "deb https://forge.glpi-project.org/attachments/download/2021/ ./" | sudo tee /etc/apt/sources.list.d/glpi-agent.list

sudo apt update
sudo apt install glpi-agent -y

# Configuration
sudo nano /etc/glpi-agent/agent.cfg
# Ajouter :
server = https://glpi.monentreprise.local/front/inventory.php
tag = Production
\end{lstlisting}

\subsection{Catégorisation des équipements}

\subsubsection{Types d'équipements gérés}
\begin{center}
\begin{tabular}{|l|l|l|}
\hline
\textbf{Catégorie} & \textbf{Sous-catégories} & \textbf{Informations suivies} \\
\hline
Ordinateurs & Portables, Fixes, Serveurs & CPU, RAM, Disque, OS \\
Périphériques & Écrans, Imprimantes, Scanners & Modèle, Résolution, Consommables \\
Réseau & Switches, Routeurs, Points d'accès & Ports, VLAN, Firmware \\
Téléphonie & Téléphones IP, Mobiles & Extensions, Forfaits \\
Logiciels & Licences, Applications & Versions, Nombre d'utilisateurs \\
\hline
\end{tabular}
\end{center}

\subsection{Processus de gestion}

\subsubsection{Cycle de vie des équipements}
\begin{enumerate}
    \item \textbf{Commande} : Création de la demande d'achat
    \item \textbf{Réception} : Enregistrement en stock
    \item \textbf{Déploiement} : Attribution à un utilisateur/service
    \item \textbf{Maintenance} : Suivi des interventions
    \item \textbf{Fin de vie} : Retrait et élimination sécurisée
\end{enumerate}

\subsubsection{Gestion des incidents}
\infobox{Workflow des tickets}{
\begin{itemize}
    \item \textbf{Création} : Automatique (utilisateur) ou manuelle (helpdesk)
    \item \textbf{Classification} : Catégorisation et priorisation
    \item \textbf{Attribution} : Assignation au technicien compétent
    \item \textbf{Résolution} : Intervention et documentation
    \item \textbf{Clôture} : Validation utilisateur et archivage
\end{itemize}
}

% ========================================
% CONFIGURATION AVANCÉE
% ========================================
\section{Configuration Avancée}

\subsection{Personnalisation de l'interface}

\subsubsection{Création des profils utilisateurs}
\begin{itemize}
    \item \textbf{Administrateur} : Accès complet à tous les modules
    \item \textbf{Technicien} : Gestion des tickets et inventaire
    \item \textbf{Utilisateur} : Création de tickets et consultation du matériel attribué
    \item \textbf{Invité} : Consultation limitée en lecture seule
\end{itemize}

\subsection{Automatisation}

\subsubsection{Règles métier}
Configuration des règles automatiques :
\begin{itemize}
    \item Attribution automatique des tickets selon les critères
    \item Escalade automatique en cas de dépassement des délais
    \item Classification automatique basée sur le contenu
    \item Notification automatique des parties prenantes
\end{itemize}

\subsection{Intégrations}

\subsubsection{Connecteurs externes}
\begin{itemize}
    \item \textbf{Active Directory/LDAP} : Synchronisation des utilisateurs
    \item \textbf{Système de supervision} : Intégration avec Nagios/Zabbix
    \item \textbf{Système de sauvegarde} : Reporting automatique
    \item \textbf{Outils de déploiement} : Intégration avec MDM/WSUS
\end{itemize}

% ========================================
% EXPLOITATION ET MAINTENANCE
% ========================================
\section{Exploitation et Maintenance}

\subsection{Monitoring et supervision}

\subsubsection{Indicateurs clés (KPI)}
\begin{center}
\begin{tabular}{|l|l|l|}
\hline
\textbf{Métrique} & \textbf{Objectif} & \textbf{Seuil d'alerte} \\
\hline
Temps de résolution moyen & < 4 heures & > 6 heures \\
Taux de résolution au 1er niveau & > 70\% & < 60\% \\
Satisfaction utilisateur & > 4/5 & < 3.5/5 \\
Disponibilité système & > 99\% & < 95\% \\
\hline
\end{tabular}
\end{center}

\subsection{Procédures de maintenance}

\subsubsection{Maintenance préventive}
\begin{itemize}
    \item \textbf{Quotidienne} : Vérification des sauvegardes et logs
    \item \textbf{Hebdomadaire} : Analyse des performances et rapports
    \item \textbf{Mensuelle} : Mise à jour des signatures et patches
    \item \textbf{Trimestrielle} : Audit complet et optimisation
\end{itemize}

\subsection{Formation des utilisateurs}

\subsubsection{Plan de formation}
\begin{enumerate}
    \item \textbf{Utilisateurs finaux} :
    \begin{itemize}
        \item Création et suivi des tickets
        \item Consultation de l'inventaire personnel
        \item Base de connaissances
    \end{itemize}
    
    \item \textbf{Techniciens} :
    \begin{itemize}
        \item Gestion avancée des tickets
        \item Maintenance de l'inventaire
        \item Reporting et statistiques
    \end{itemize}
    
    \item \textbf{Administrateurs} :
    \begin{itemize}
        \item Configuration système
        \item Gestion des droits et profils
        \item Optimisation et évolutions
    \end{itemize}
\end{enumerate}

% ========================================
% RÉSULTATS ET BILAN
% ========================================
\section{Résultats et Bilan}

\subsection{Objectifs atteints}

\successbox{Réussites du projet}{
\begin{itemize}
    \item Installation réussie de GLPI en environnement de production
    \item Inventaire automatisé de [X] équipements
    \item Formation de [Y] utilisateurs
    \item Mise en place d'un processus de gestion des incidents
    \item Amélioration de la traçabilité du parc informatique
\end{itemize}
}

\subsection{Métriques de performance}

\subsubsection{Avant/Après mise en place}
\begin{center}
\begin{tabular}{|l|l|l|l|}
\hline
\textbf{Indicateur} & \textbf{Avant} & \textbf{Après} & \textbf{Amélioration} \\
\hline
Temps de traitement incident & 2 jours & 4 heures & -83\% \\
Taux d'inventaire à jour & 30\% & 95\% & +217\% \\
Satisfaction utilisateur & 2.5/5 & 4.2/5 & +68\% \\
Coût de gestion/équipement & 15€/mois & 8€/mois & -47\% \\
\hline
\end{tabular}
\end{center}

\subsection{Difficultés rencontrées}

\warningbox{Points d'attention}{
\begin{itemize}
    \item \textbf{Résistance au changement} : Formation approfondie nécessaire
    \item \textbf{Qualité des données} : Nettoyage initial important
    \item \textbf{Intégrations} : Complexité des connecteurs externes
    \item \textbf{Performance} : Optimisation base de données requise
\end{itemize}
}

% ========================================
% ÉVOLUTIONS ET PERSPECTIVES
% ========================================
\section{Évolutions et Perspectives}

\subsection{Améliorations prévues}

\subsubsection{Court terme (3 mois)}
\begin{itemize}
    \item Déploiement de l'agent sur tous les postes
    \item Intégration avec l'Active Directory
    \item Mise en place des SLA automatisés
    \item Formation complémentaire des utilisateurs
\end{itemize}

\subsubsection{Moyen terme (6-12 mois)}
\begin{itemize}
    \item Intégration avec les outils de supervision
    \item Mise en place d'un portail self-service avancé
    \item Automatisation des processus de déploiement
    \item Reporting avancé et tableaux de bord
\end{itemize}

\subsection{ROI et bénéfices}

\subsubsection{Gains quantifiables}
\begin{itemize}
    \item \textbf{Gain de temps} : 2h/jour/technicien économisées
    \item \textbf{Réduction des coûts} : -30\% sur la gestion du parc
    \item \textbf{Amélioration SLA} : Respect des délais +85\%
    \item \textbf{Optimisation achats} : Planification basée sur les données réelles
\end{itemize}

% ========================================
% CONCLUSION
% ========================================
\section{Conclusion}

La mise en place de GLPI pour la gestion du patrimoine informatique s'est révélée être un succès notable. L'installation et la configuration de la solution ont permis d'établir un système robuste et évolutif, capable de répondre aux besoins actuels et futurs de l'organisation.

Les principaux acquis de ce projet sont :
\begin{itemize}
    \item Une maîtrise technique complète de l'installation et de la configuration de GLPI
    \item La mise en place d'un processus structuré de gestion du parc informatique
    \item L'amélioration significative de la qualité de service IT
    \item Le développement de compétences en administration système et base de données
\end{itemize}

Ce projet m'a permis d'approfondir mes connaissances en :
\begin{itemize}
    \item Administration de serveurs Linux
    \item Gestion de bases de données MySQL
    \item Configuration de serveurs web Apache
    \item Sécurisation d'applications web
    \item Gestion de projet IT
\end{itemize}

L'expérience acquise lors de ce projet constitue un atout précieux pour mon parcours professionnel en tant que futur technicien en services informatiques aux organisations.

% ========================================
% ANNEXES
% ========================================
\newpage
\section{Annexes}

\subsection{Annexe A : Guide utilisateur}

\infobox{Procédure pour les utilisateurs}{
\textbf{Étapes pour créer un ticket d'incident :}
\begin{enumerate}
    \item Se connecter à GLPI avec ses identifiants
    \item Cliquer sur "Créer un ticket" dans le menu principal
    \item Remplir les champs obligatoires :
    \begin{itemize}
        \item Titre : Description courte du problème
        \item Catégorie : Sélectionner le type d'incident
        \item Urgence : Évaluer l'impact sur votre travail
        \item Description : Détailler le problème rencontré
    \end{itemize}
    \item Joindre des captures d'écran si nécessaire
    \item Valider la création du ticket
    \item Noter le numéro de ticket pour le suivi
\end{enumerate}
}

\subsection{Annexe B : Références et sources}

\begin{itemize}
    \item Documentation officielle GLPI : \url{https://glpi-project.org/documentation/}
    \item Guide d'installation : \url{https://glpi-install.readthedocs.io/}
    \item Forum communautaire : \url{https://forum.glpi-project.org/}
    \item Dépôt GitHub : \url{https://github.com/glpi-project/glpi}
    \item Best practices ITIL : \url{https://www.axelos.com/best-practice-solutions/itil}
\end{itemize}

\end{document}