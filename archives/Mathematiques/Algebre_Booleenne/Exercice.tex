\documentclass{article}
\usepackage{circuitikz}
\usepackage{amsmath}
\begin{document}

	extbf{Énoncé :}

On dispose de 3 alarmes $a$, $b$, $c$ (valeurs logiques 0 ou 1).

\begin{itemize}
  \item Allumer une lampe $L$ s'il y a au moins une alarme active.
  \item Déclencher une sonnerie $S$ s'il y a au moins deux alarmes actives.
\end{itemize}

	extbf{Table de vérité :}
\begin{center}
\begin{tabular}{|c|c|c||c|c|}
\hline
a & b & c & L & S \\
\hline
0 & 0 & 0 & 0 & 0 \\
0 & 0 & 1 & 1 & 0 \\
0 & 1 & 0 & 1 & 0 \\
0 & 1 & 1 & 1 & 1 \\
1 & 0 & 0 & 1 & 0 \\
1 & 0 & 1 & 1 & 1 \\
1 & 1 & 0 & 1 & 1 \\
1 & 1 & 1 & 1 & 1 \\
\hline
\end{tabular}
\end{center}

	extbf{Expressions logiques :}
\begin{align*}
L &= a \vee b \vee c \\
S &= (a \wedge b) \vee (a \wedge c) \vee (b \wedge c)
\end{align*}

	extbf{Logigramme logique (schéma) :}
\begin{center}
\begin{circuitikz}[american]
  % Entrées
  \node (a) at (0,2) {$a$};
  \node (b) at (0,1) {$b$};
  \node (c) at (0,0) {$c$};
  % OU pour la lampe
  \node[or port, right=2 of a] (or1) {};
  \draw (a) -- (or1.in 1);
  \draw (b) -- ++(0.7,0) |- (or1.in 2);
  \draw (c) -- ++(0.7,0) |- (or1.in 3);
  \draw (or1.out) -- ++(1,0) node[right] {L};
  % ET pour la sonnerie
  \node[and port, right=2 of c] (and1) {};
  \node[and port, right=2 of b] (and2) {};
  \node[and port, right=2 of a] (and3) {};
  % a ET b
  \draw (a) -- ++(0.7,0) |- (and1.in 1);
  \draw (b) -- ++(0.7,0) |- (and1.in 2);
  % a ET c
  \draw (a) -- ++(0.7,0) |- (and2.in 1);
  \draw (c) -- ++(0.7,0) |- (and2.in 2);
  % b ET c
  \draw (b) -- ++(0.7,0) |- (and3.in 1);
  \draw (c) -- ++(0.7,0) |- (and3.in 2);
  % OU final pour S
  \node[or port, right=2.5 of and2] (or2) {};
  \draw (and1.out) -- ++(0.5,0) |- (or2.in 1);
  \draw (and2.out) -- ++(0.5,0) |- (or2.in 2);
  \draw (and3.out) -- ++(0.5,0) |- (or2.in 3);
  \draw (or2.out) -- ++(1,0) node[right] {S};
\end{circuitikz}
\end{center}

\end{document}
