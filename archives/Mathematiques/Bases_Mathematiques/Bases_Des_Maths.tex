\documentclass{article}
\usepackage[utf8]{inputenc}
\usepackage[T1]{fontenc}
\usepackage[french]{babel}
\usepackage{geometry}
\usepackage{amsmath}
\geometry{margin=2cm}

\title{Bases des Mathématiques}
\author{Mignard Mael}
\date{\today}

\begin{document}

\maketitle

\section{Division euclidienne}
La division euclidienne est une opération fondamentale en mathématiques qui consiste à diviser un entier par un autre entier non nul, en obtenant un quotient et un reste. Plus formellement, pour deux entiers \( a \) et \( b \) (avec \( b \neq 0 \)), il existe des entiers uniques \( q \) (le quotient) et \( r \) (le reste) tels que :
$ a = bq + r $
où \( 0 \leq r < |b| \).

\subsection{Exemple}
Par exemple, si nous divisons 17 par 5, nous avons :
\[ 17 = 5 \times 3 + 2
\]
Ici, le quotient \( q \) est 3 et le reste \( r \) est 2.
\[ 18 = 6 \times 3 + 0
\]
Ici, le quotient \( q \) est 3 et le reste \( r \) est 0.

\subsection{Exercices}
\subsubsection{Exercice 1}
Effectuer les divisions euclidiennes suivantes et indiquer le quotient et le reste :
 \\
$
128 \ / \ 12 =  \ \  (10  \ reste : \ 8 )\\
$
\\
$
37 \ / \ 5 =  \ \ (7 \times 7 +2)\\
$
\\
$
256  \ / \  17 =  \ \ (15 \times 17 +1) \\
$
\\
$999 \ / \ 25 =  \ \ (39 \times 25 +24) \\ 
$
\\
\subsubsection{Exercice 2}
(Exercice des pages avec le nombres de lettres)

1 : La 10000ème lettre sera un P, en tout 384 alphabets complets, puis 16 lettres dans le 385ème alphabet. seront écrits. 
\\
$10000 = 26 \times 284 + 16$  \\
\\ 
2 : La derniere page contiendra 53 lignes. \\
$6245 = 72 \times 86 + 53$ \\
\\
\subsubsection{Exercice 3}

Vrai ou Faux ? \\
\\
$5 \mathbin{\left|\,\right.} 45 \ Vrai$ \\ \\
$7 \mathbin{\left|\,\right.} 50 \ Faux$ \\ \\
$12 \mathbin{\left|\,\right.} 144 \ Vrai$ \\ \\
$8 \mathbin{\left|\,\right.} 260 \ Faux$ \\ \\
\end{document}