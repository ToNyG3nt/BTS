\documentclass[12pt,a4paper]{article}
\usepackage[utf8]{inputenc}
\usepackage[T1]{fontenc}
\usepackage[french]{babel}
\usepackage{geometry}
\geometry{margin=2.5cm}
\title{Découverte de la programmation : TP1 corrigé logigrammes et pseudo-code}
\author{Maël Mignard}
\date{\today}

\begin{document}
\maketitle

\section*{Exercice 1}
\textbf{Pseudo-code corrigé}
\begin{verbatim}
Saisir nom
Si longueur(nom) > 20 alors
    Afficher "Votre nom est trop long"
Sinon
    Afficher "Votre nom est trop court"
FinSi
\end{verbatim}
\textbf{Explication :} On teste si la longueur du nom dépasse 20 caractères. Si oui, on affiche le message correspondant, sinon l'autre.

\section*{Exercice 2}
\textbf{Pseudo-code corrigé}
\begin{verbatim}
Saisir nom
Si longueur(nom) > 30 alors
    Afficher "Votre nom est trop long"
Sinon Si longueur(nom) < 30 alors
    Afficher "Votre nom est trop court"
FinSi
\end{verbatim}
\textbf{Explication :} On distingue bien les deux cas (strictement supérieur ou inférieur à 30). Si la longueur vaut exactement 30, aucun message n'est affiché.

\section*{Exercice 3}
\textbf{Pseudo-code corrigé}
\begin{verbatim}
Pour i allant de 1 à 10
    Afficher "Je suis au tour n°", i
FinPour
\end{verbatim}
\textbf{Explication :} On utilise une boucle pour afficher le message pour chaque valeur de i de 1 à 10.

\section*{Exercice 5}
\textbf{a) Algorithme (pseudo-code)}
\begin{verbatim}
Répéter
    Saisir login
    Saisir motDePasse
    resultat <- testerUtilisateur(login, motDePasse)
Jusqu'à resultat = 1
\end{verbatim}

\textbf{b) Avec messages}
\begin{verbatim}
Répéter
    Saisir login
    Saisir motDePasse
    resultat <- testerUtilisateur(login, motDePasse)
    Si resultat = 1 alors
        Afficher "identifiants corrects"
    Sinon
        Afficher "identifiants incorrects"
    FinSi
Jusqu'à resultat = 1
\end{verbatim}

\textbf{c) Limite à 3 tentatives}
\begin{verbatim}
tentative <- 0
Répéter
    Saisir login
    Saisir motDePasse
    resultat <- testerUtilisateur(login, motDePasse)
    tentative <- tentative + 1
    Si resultat = 1 alors
        Afficher "identifiants corrects"
    Sinon
        Afficher "identifiants incorrects"
    FinSi
Jusqu'à resultat = 1 OU tentative = 3
\end{verbatim}

\textbf{d) Message blocage après 3 essais}
\begin{verbatim}
tentative <- 0
Répéter
    Saisir login
    Saisir motDePasse
    resultat <- testerUtilisateur(login, motDePasse)
    tentative <- tentative + 1
    Si resultat = 1 alors
        Afficher "identifiants corrects"
    Sinon
        Afficher "identifiants incorrects"
    FinSi
Jusqu'à resultat = 1 OU tentative = 3
Si resultat <> 1 alors
    Afficher "Vous ne pouvez plus vous connecter"
FinSi
\end{verbatim}

\section*{Exercice 6}
\textbf{a) Algorithme (pseudo-code)}
\begin{verbatim}
Saisir montantHT
TTC <- montantHT * 1.2
Afficher TTC
\end{verbatim}

\textbf{b) Avec vérification numérique}
\begin{verbatim}
Saisir montantHT
Si isNumerique(montantHT) = 1 alors
    TTC <- montantHT * 1.2
    Afficher TTC
Sinon
    Afficher "Vous n'avez pas saisi un montant valide"
FinSi
\end{verbatim}

\textbf{c) Redemander tant que non numérique}
\begin{verbatim}
Répéter
    Saisir montantHT
    Si isNumerique(montantHT) = 0 alors
        Afficher "Vous n'avez pas saisi un montant valide"
    FinSi
Jusqu'à isNumerique(montantHT) = 1
TTC <- montantHT * 1.2
Afficher TTC
\end{verbatim}

\end{document}
