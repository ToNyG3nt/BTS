\documentclass[12pt,a4paper]{article}
\usepackage[utf8]{inputenc}
\usepackage[T1]{fontenc}
\usepackage[french]{babel}
\usepackage{geometry}
\usepackage{tikz}
\geometry{margin=2.5cm}
\title{Découverte de la programmation : logigrammes et pseudo-code}
\author{Maël Mignard}
\date{\today}

\begin{document}
\maketitle

\section*{Exercice 1}
\textbf{Énoncé :} On saisit un nom d’utilisateur :
\begin{itemize}
  \item si sa longueur est strictement supérieure à 20, on affiche le message « Votre nom est trop long »,
  \item sinon on affiche le message « Votre nom est trop court ».
\end{itemize}

\subsection*{Logigramme}
\begin{center}
\begin{tikzpicture}[node distance=2cm, every node/.style={font=\sffamily}]
  \node (start) [draw, rectangle, rounded corners] {Début};
  \node (input) [draw, rectangle, below of=start] {Saisir nom};
  \node (test) [draw, diamond, aspect=2, below of=input, yshift=-0.5cm] {longueur(nom) $>$ 20 ?};
  \node (long) [draw, rectangle, right=3cm of test] {Afficher « Votre nom est trop long »};
  \node (court) [draw, rectangle, below of=test, yshift=-0.5cm] {Afficher « Votre nom est trop court »};
  \node (end) [draw, rectangle, rounded corners, below of=court] {Fin};
  \draw[->] (start) -- (input);
  \draw[->] (input) -- (test);
  \draw[->] (test) -- node[above] {Oui} (long);
  \draw[->] (test) -- node[left] {Non} (court);
  \draw[->] (long) |- (end);
  \draw[->] (court) -- (end);
\end{tikzpicture}
\end{center}

\subsection*{Pseudo-code}
\begin{verbatim}
Saisir nom
Si longueur(nom) > 20 alors
    Afficher "Votre nom est trop long"
Sinon
    Afficher "Votre nom est trop court"
FinSi
\end{verbatim}

\section*{Exercice 2}
\textbf{Énoncé :} On saisit un nom d’utilisateur et :
\begin{itemize}
  \item si sa longueur est strictement supérieure à 30, on affiche le message « Votre nom est trop long »,
  \item si sa longueur est strictement inférieure à 30, on affiche le message « Votre nom est trop court ».
\end{itemize}

\subsection*{Logigramme}
\begin{center}
\begin{tikzpicture}[node distance=2cm, every node/.style={font=\sffamily}]
  \node (start) [draw, rectangle, rounded corners] {Début};
  \node (input) [draw, rectangle, below of=start] {Saisir nom};
  \node (test1) [draw, diamond, aspect=2, below of=input, yshift=-0.5cm] {longueur(nom) $>$ 30 ?};
  \node (long) [draw, rectangle, right=3cm of test1] {Afficher « Votre nom est trop long »};
  \node (test2) [draw, diamond, aspect=2, below of=test1, yshift=-0.5cm] {longueur(nom) $<$ 30 ?};
  \node (court) [draw, rectangle, right=3cm of test2] {Afficher « Votre nom est trop court »};
  \node (end) [draw, rectangle, rounded corners, below of=test2, yshift=-1cm] {Fin};
  \draw[->] (start) -- (input);
  \draw[->] (input) -- (test1);
  \draw[->] (test1) -- node[above] {Oui} (long);
  \draw[->] (test1) -- node[left] {Non} (test2);
  \draw[->] (long) |- (end);
  \draw[->] (test2) -- node[above] {Oui} (court);
  \draw[->] (test2) -- node[left] {Non} (end);
  \draw[->] (court) |- (end);
\end{tikzpicture}
\end{center}

\subsection*{Pseudo-code}
\begin{verbatim}
Saisir nom
Si longueur(nom) > 30 alors
    Afficher "Votre nom est trop long"
Sinon Si longueur(nom) < 30 alors
    Afficher "Votre nom est trop court"
FinSi
\end{verbatim}

\section*{Exercice 3}
\textbf{Énoncé :}
On souhaite afficher le message suivant à l’écran :
« Je suis au tour n°1
Je suis au tour n°2
Je suis au tour …
…etc. …..
Je suis au tour n° 10 »

\subsection*{Logigramme}
\begin{center}
\begin{tikzpicture}[node distance=2cm, every node/.style={font=\sffamily}]
  \node (start) [draw, rectangle, rounded corners] {Début};
  \node (init) [draw, rectangle, below of=start] {i $\leftarrow$ 1};
  \node (test) [draw, diamond, aspect=2, below of=init, yshift=-0.5cm] {i $\leq$ 10 ?};
  \node (aff) [draw, rectangle, right=3cm of test] {Afficher « Je suis au tour n°i »};
  \node (inc) [draw, rectangle, below of=aff] {i $\leftarrow$ i+1};
  \node (end) [draw, rectangle, rounded corners, below of=test, yshift=-2cm] {Fin};
  \draw[->] (start) -- (init);
  \draw[->] (init) -- (test);
  \draw[->] (test) -- node[above] {Oui} (aff);
  \draw[->] (aff) -- (inc);
  \draw[->] (inc) -| (test);
  \draw[->] (test) -- node[left] {Non} (end);
\end{tikzpicture}
\end{center}

\subsection*{Pseudo-code}
\begin{verbatim}
Pour i allant de 1 à 10
    Afficher "Je suis au tour n°", i
FinPour
\end{verbatim}

\end{document}
