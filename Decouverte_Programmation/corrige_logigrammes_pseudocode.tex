\documentclass[12pt,a4paper]{article}
\usepackage[utf8]{inputenc}
\usepackage[T1]{fontenc}
\usepackage[french]{babel}
\usepackage{geometry}
\geometry{margin=2.5cm}
\title{Découverte de la programmation : corrigé logigrammes et pseudo-code}
\author{Maël Mignard}
\date{\today}

\begin{document}
\maketitle

\section*{Correction Exercice 1}
\textbf{Pseudo-code corrigé}
\begin{verbatim}
Saisir nom
Si longueur(nom) > 20 alors
    Afficher "Votre nom est trop long"
Sinon
    Afficher "Votre nom est trop court"
FinSi
\end{verbatim}
\textbf{Explication :} On teste si la longueur du nom dépasse 20 caractères. Si oui, on affiche le message correspondant, sinon l'autre.

\section*{Correction Exercice 2}
\textbf{Pseudo-code corrigé}
\begin{verbatim}
Saisir nom
Si longueur(nom) > 30 alors
    Afficher "Votre nom est trop long"
Sinon Si longueur(nom) < 30 alors
    Afficher "Votre nom est trop court"
FinSi
\end{verbatim}
\textbf{Explication :} On distingue bien les deux cas (strictement supérieur ou inférieur à 30). Si la longueur vaut exactement 30, aucun message n'est affiché.

\section*{Correction Exercice 3}
\textbf{Pseudo-code corrigé}
\begin{verbatim}
Pour i allant de 1 à 10
    Afficher "Je suis au tour n°", i
FinPour
\end{verbatim}
\textbf{Explication :} On utilise une boucle pour afficher le message pour chaque valeur de i de 1 à 10.

\end{document}
