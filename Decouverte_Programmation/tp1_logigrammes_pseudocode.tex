\documentclass[12pt,a4paper]{article}
\usepackage[utf8]{inputenc}
\usepackage[T1]{fontenc}
\usepackage[french]{babel}
\usepackage{geometry}
\usepackage{tikz}
\usetikzlibrary{positioning}

\geometry{margin=2.5cm}
\title{Découverte de la programmation : TP1 logigrammes et pseudo-code}
\author{Maël Mignard}
\date{\today}

\begin{document}
\maketitle

% Exercice 1
\section*{Exercice 1}
\textbf{Énoncé :} On saisit un nom d’utilisateur :
\begin{itemize}
  \item si sa longueur est strictement supérieure à 20, on affiche le message « Votre nom est trop long »,
  \item sinon on affiche le message « Votre nom est trop court ».
\end{itemize}

\subsection*{Logigramme}
\begin{center}
\begin{tikzpicture}[every node/.style={font=\sffamily}, node distance=1.7cm, >=latex]
  \node (start) [draw, rectangle, rounded corners, minimum width=2.5cm, minimum height=0.8cm] {Début};
  \node (input) [draw, rectangle, below=of start, minimum width=2.5cm, minimum height=0.8cm] {Saisir nom};
  \node (test) [draw, diamond, aspect=2, below=of input, inner sep=1pt, minimum width=3.5cm, minimum height=1.2cm] {longueur(nom) $>$ 20 ?};
  \node (long) [draw, rectangle, right=3.5cm of test, minimum width=2.8cm, minimum height=0.8cm] {Afficher « Votre nom est trop long »};
  \node (court) [draw, rectangle, below=of test, minimum width=2.8cm, minimum height=0.8cm] {Afficher « Votre nom est trop court »};
  \node (end) [draw, rectangle, rounded corners, below=of court, minimum width=2.5cm, minimum height=0.8cm] {Fin};
  \draw[->] (start) -- (input);
  \draw[->] (input) -- (test);
  \draw[->] (test) -- node[above, sloped] {Oui} (long);
  \draw[->] (test) -- node[left] {Non} (court);
  \draw[->] (long) |- (end);
  \draw[->] (court) -- (end);
\end{tikzpicture}
\end{center}

\subsection*{Pseudo-code}
\begin{verbatim}
Saisir nom
Si longueur(nom) > 20 alors
    Afficher "Votre nom est trop long"
Sinon
    Afficher "Votre nom est trop court"
FinSi
\end{verbatim}

% Exercice 2
\section*{Exercice 2}
\textbf{Énoncé :} On saisit un nom d’utilisateur et :
\begin{itemize}
  \item si sa longueur est strictement supérieure à 30, on affiche le message « Votre nom est trop long »,
  \item si sa longueur est strictement inférieure à 30, on affiche le message « Votre nom est trop court ».
\end{itemize}

\subsection*{Logigramme}
\begin{center}
\begin{tikzpicture}[every node/.style={font=\sffamily}, node distance=1.7cm, >=latex]
  \node (start) [draw, rectangle, rounded corners, minimum width=2.5cm, minimum height=0.8cm] {Début};
  \node (input) [draw, rectangle, below=of start, minimum width=2.5cm, minimum height=0.8cm] {Saisir nom};
  \node (test1) [draw, diamond, aspect=2, below=of input, inner sep=1pt, minimum width=3.5cm, minimum height=1.2cm] {longueur(nom) $>$ 30 ?};
  \node (long) [draw, rectangle, right=3.5cm of test1, minimum width=2.8cm, minimum height=0.8cm] {Afficher « Votre nom est trop long »};
  \node (test2) [draw, diamond, aspect=2, below=of test1, inner sep=1pt, minimum width=3.5cm, minimum height=1.2cm] {longueur(nom) $<$ 30 ?};
  \node (court) [draw, rectangle, right=3.5cm of test2, minimum width=2.8cm, minimum height=0.8cm] {Afficher « Votre nom est trop court »};
  \node (end) [draw, rectangle, rounded corners, below=of test2, minimum width=2.5cm, minimum height=0.8cm] {Fin};
  \draw[->] (start) -- (input);
  \draw[->] (input) -- (test1);
  \draw[->] (test1) -- node[above, sloped] {Oui} (long);
  \draw[->] (test1) -- node[left] {Non} (test2);
  \draw[->] (long) |- (end);
  \draw[->] (test2) -- node[above, sloped] {Oui} (court);
  \draw[->] (test2) -- node[left] {Non} (end);
  \draw[->] (court) |- (end);
\end{tikzpicture}
\end{center}

\subsection*{Pseudo-code}
\begin{verbatim}
Saisir nom
Si longueur(nom) > 30 alors
    Afficher "Votre nom est trop long"
Sinon Si longueur(nom) < 30 alors
    Afficher "Votre nom est trop court"
FinSi
\end{verbatim}

% Exercice 3
\section*{Exercice 3}
\textbf{Énoncé :}
On souhaite afficher le message suivant à l’écran :
« Je suis au tour n°1
Je suis au tour n°2
Je suis au tour …
…etc. …..
Je suis au tour n° 10 »

\subsection*{Logigramme}
\begin{center}
\begin{tikzpicture}[every node/.style={font=\sffamily}, node distance=1.7cm, >=latex]
  \node (start) [draw, rectangle, rounded corners, minimum width=2.5cm, minimum height=0.8cm] {Début};
  \node (init) [draw, rectangle, below=of start, minimum width=2.5cm, minimum height=0.8cm] {i $\leftarrow$ 1};
  \node (test) [draw, diamond, aspect=2, below=of init, inner sep=1pt, minimum width=3.5cm, minimum height=1.2cm] {i $\leq$ 10 ?};
  \node (aff) [draw, rectangle, right=3.5cm of test, minimum width=3.2cm, minimum height=0.8cm] {Afficher « Je suis au tour n°i »};
  \node (inc) [draw, rectangle, below=of aff, minimum width=2.5cm, minimum height=0.8cm] {i $\leftarrow$ i+1};
  \node (end) [draw, rectangle, rounded corners, below=of test, minimum width=2.5cm, minimum height=0.8cm] {Fin};
  \draw[->] (start) -- (init);
  \draw[->] (init) -- (test);
  \draw[->] (test) -- node[above, sloped] {Oui} (aff);
  \draw[->] (aff) -- (inc);
  \draw[->] (inc) -| (test);
  \draw[->] (test) -- node[left] {Non} (end);
\end{tikzpicture}
\end{center}

\subsection*{Pseudo-code}
\begin{verbatim}
Pour i allant de 1 à 10
    Afficher "Je suis au tour n°", i
FinPour
\end{verbatim}

% Exercice 5
\section*{Exercice 5}
\textbf{a) Algorithme (pseudo-code)}
\begin{verbatim}
Répéter
    Saisir login
    Saisir motDePasse
    resultat <- testerUtilisateur(login, motDePasse)
Jusqu'à resultat = 1
\end{verbatim}

\textbf{b) Avec messages}
\begin{verbatim}
Répéter
    Saisir login
    Saisir motDePasse
    resultat <- testerUtilisateur(login, motDePasse)
    Si resultat = 1 alors
        Afficher "identifiants corrects"
    Sinon
        Afficher "identifiants incorrects"
    FinSi
Jusqu'à resultat = 1
\end{verbatim}

\textbf{c) Limite à 3 tentatives}
\begin{verbatim}
tentative <- 0
Répéter
    Saisir login
    Saisir motDePasse
    resultat <- testerUtilisateur(login, motDePasse)
    tentative <- tentative + 1
    Si resultat = 1 alors
        Afficher "identifiants corrects"
    Sinon
        Afficher "identifiants incorrects"
    FinSi
Jusqu'à resultat = 1 OU tentative = 3
\end{verbatim}

\textbf{d) Message blocage après 3 essais}
\begin{verbatim}
tentative <- 0
Répéter
    Saisir login
    Saisir motDePasse
    resultat <- testerUtilisateur(login, motDePasse)
    tentative <- tentative + 1
    Si resultat = 1 alors
        Afficher "identifiants corrects"
    Sinon
        Afficher "identifiants incorrects"
    FinSi
Jusqu'à resultat = 1 OU tentative = 3
Si resultat <> 1 alors
    Afficher "Vous ne pouvez plus vous connecter"
FinSi
\end{verbatim}

% Exercice 6
\section*{Exercice 6}
\textbf{a) Algorithme (pseudo-code)}
\begin{verbatim}
Saisir montantHT
TTC <- montantHT * 1.2
Afficher TTC
\end{verbatim}

\textbf{b) Avec vérification numérique}
\begin{verbatim}
Saisir montantHT
Si isNumerique(montantHT) = 1 alors
    TTC <- montantHT * 1.2
    Afficher TTC
Sinon
    Afficher "Vous n'avez pas saisi un montant valide"
FinSi
\end{verbatim}

\textbf{c) Redemander tant que non numérique}
\begin{verbatim}
Répéter
    Saisir montantHT
    Si isNumerique(montantHT) = 0 alors
        Afficher "Vous n'avez pas saisi un montant valide"
    FinSi
Jusqu'à isNumerique(montantHT) = 1
TTC <- montantHT * 1.2
Afficher TTC
\end{verbatim}

% Exercice 7

\section*{Exercice 7}

\subsection*{Logigramme}
\begin{center}
\begin{tikzpicture}[every node/.style={font=\sffamily}, node distance=1.7cm, >=latex]
  \node (start) [draw, rectangle, rounded corners, minimum width=2.5cm, minimum height=0.8cm] {Début};
  \node (in1) [draw, rectangle, below=of start, minimum width=2.5cm, minimum height=0.8cm] {Saisir phrase1};
  \node (in2) [draw, rectangle, below=of in1, minimum width=2.5cm, minimum height=0.8cm] {Saisir phrase2};
  \node (in3) [draw, rectangle, below=of in2, minimum width=2.5cm, minimum height=0.8cm] {Saisir phrase3};
  \node (concat) [draw, rectangle, below=of in3, minimum width=4.5cm, minimum height=0.8cm] {concat $\leftarrow$ phrase1 + " " + phrase2 + " " + phrase3};
  \node (aff) [draw, rectangle, below=of concat, minimum width=2.5cm, minimum height=0.8cm] {Afficher concat};
  \node (end) [draw, rectangle, rounded corners, below=of aff, minimum width=2.5cm, minimum height=0.8cm] {Fin};
  \draw[->] (start) -- (in1);
  \draw[->] (in1) -- (in2);
  \draw[->] (in2) -- (in3);
  \draw[->] (in3) -- (concat);
  \draw[->] (concat) -- (aff);
  \draw[->] (aff) -- (end);
\end{tikzpicture}
\end{center}

\subsection*{Pseudo-code}
\begin{verbatim}
Saisir phrase1
Saisir phrase2
Saisir phrase3
concat <- phrase1 + " " + phrase2 + " " + phrase3
Afficher concat
\end{verbatim}

	extbf{b) Saisir 3 phrases, chaque phrase max 50 caractères}

\subsection*{Logigramme}
\begin{center}
\begin{tikzpicture}[every node/.style={font=\sffamily}, node distance=1.7cm, >=latex]
  \node (start) [draw, rectangle, rounded corners, minimum width=2.5cm, minimum height=0.8cm] {Début};
  \node (i) [draw, rectangle, below=of start, minimum width=2.5cm, minimum height=0.8cm] {i $\leftarrow$ 1};
  \node (testi) [draw, diamond, aspect=2, below=of i, inner sep=1pt, minimum width=3.5cm, minimum height=1.2cm] {i $\leq$ 3 ?};
  \node (repeat) [draw, rectangle, right=3.5cm of testi, minimum width=3.5cm, minimum height=0.8cm] {Saisir phrase[i]};
  \node (testlen) [draw, diamond, aspect=2, below=of repeat, inner sep=1pt, minimum width=3.5cm, minimum height=1.2cm] {longueur(phrase[i]) $\leq$ 50 ?};
  \node (inc) [draw, rectangle, below=of testi, minimum width=2.5cm, minimum height=0.8cm] {i $\leftarrow$ i+1};
  \node (concat) [draw, rectangle, below=of inc, minimum width=4.5cm, minimum height=0.8cm] {concat $\leftarrow$ phrase[1] + " " + phrase[2] + " " + phrase[3]};
  \node (aff) [draw, rectangle, below=of concat, minimum width=2.5cm, minimum height=0.8cm] {Afficher concat};
  \node (end) [draw, rectangle, rounded corners, below=of aff, minimum width=2.5cm, minimum height=0.8cm] {Fin};
  \draw[->] (start) -- (i);
  \draw[->] (i) -- (testi);
  \draw[->] (testi) -- node[above, sloped] {Oui} (repeat);
  \draw[->] (repeat) -- (testlen);
  \draw[->] (testlen) -- node[right] {Oui} (inc);
  \draw[->] (testlen) -- node[right] {Non} (repeat);
  \draw[->] (testi) -- node[left] {Non} (concat);
  \draw[->] (inc) -- (testi);
  \draw[->] (concat) -- (aff);
  \draw[->] (aff) -- (end);
\end{tikzpicture}
\end{center}

\subsection*{Pseudo-code}
\begin{verbatim}
Pour i de 1 à 3
    Répéter
        Saisir phrase[i]
    Jusqu'à longueur(phrase[i]) <= 50
FinPour
concat <- phrase[1] + " " + phrase[2] + " " + phrase[3]
Afficher concat
\end{verbatim}


\subsection*{Logigramme}
\begin{center}
\begin{tikzpicture}[every node/.style={font=\sffamily}, node distance=1.7cm, >=latex]
  \node (start) [draw, rectangle, rounded corners, minimum width=2.5cm, minimum height=0.8cm] {Début};
  \node (n) [draw, rectangle, below=of start, minimum width=2.5cm, minimum height=0.8cm] {Saisir N};
  \node (i) [draw, rectangle, below=of n, minimum width=2.5cm, minimum height=0.8cm] {i $\leftarrow$ 1};
  \node (testi) [draw, diamond, aspect=2, below=of i, inner sep=1pt, minimum width=3.5cm, minimum height=1.2cm] {i $\leq$ N ?};
  \node (repeat) [draw, rectangle, right=3.5cm of testi, minimum width=3.5cm, minimum height=0.8cm] {Saisir phrase[i]};
  \node (testlen) [draw, diamond, aspect=2, below=of repeat, inner sep=1pt, minimum width=3.5cm, minimum height=1.2cm] {longueur(phrase[i]) $\leq$ 50 ?};
  \node (inc) [draw, rectangle, below=of testi, minimum width=2.5cm, minimum height=0.8cm] {i $\leftarrow$ i+1};
  \node (concat) [draw, rectangle, below=of inc, minimum width=4.5cm, minimum height=0.8cm] {concat $\leftarrow$ phrase[1]};
  \node (i2) [draw, rectangle, below=of concat, minimum width=2.5cm, minimum height=0.8cm] {i $\leftarrow$ 2};
  \node (testi2) [draw, diamond, aspect=2, below=of i2, inner sep=1pt, minimum width=3.5cm, minimum height=1.2cm] {i $\leq$ N ?};
  \node (concat2) [draw, rectangle, right=3.5cm of testi2, minimum width=4.5cm, minimum height=0.8cm] {concat $\leftarrow$ concat + " " + phrase[i]};
  \node (inc2) [draw, rectangle, below=of testi2, minimum width=2.5cm, minimum height=0.8cm] {i $\leftarrow$ i+1};
  \node (aff) [draw, rectangle, below=of inc2, minimum width=2.5cm, minimum height=0.8cm] {Afficher concat};
  \node (end) [draw, rectangle, rounded corners, below=of aff, minimum width=2.5cm, minimum height=0.8cm] {Fin};
  \draw[->] (start) -- (n);
  \draw[->] (n) -- (i);
  \draw[->] (i) -- (testi);
  \draw[->] (testi) -- node[above, sloped] {Oui} (repeat);
  \draw[->] (repeat) -- (testlen);
  \draw[->] (testlen) -- node[right] {Oui} (inc);
  \draw[->] (testlen) -- node[right] {Non} (repeat);
  \draw[->] (testi) -- node[left] {Non} (concat);
  \draw[->] (inc) -- (testi);
  \draw[->] (concat) -- (i2);
  \draw[->] (i2) -- (testi2);
  \draw[->] (testi2) -- node[above, sloped] {Oui} (concat2);
  \draw[->] (concat2) -- (inc2);
  \draw[->] (inc2) -- (testi2);
  \draw[->] (testi2) -- node[left] {Non} (aff);
  \draw[->] (aff) -- (end);
\end{tikzpicture}
\end{center}

\subsection*{Pseudo-code}
\begin{verbatim}
Saisir N
Pour i de 1 à N
    Répéter
        Saisir phrase[i]
    Jusqu'à longueur(phrase[i]) <= 50
FinPour
concat <- phrase[1]
Pour i de 2 à N
    concat <- concat + " " + phrase[i]
FinPour
Afficher concat
\end{verbatim}

\end{document}
