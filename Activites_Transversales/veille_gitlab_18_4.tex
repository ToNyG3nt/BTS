\documentclass[12pt,a4paper]{article}
\usepackage[utf8]{inputenc}
\usepackage[T1]{fontenc}
\usepackage[french]{babel}
\usepackage{geometry}
\geometry{margin=2.5cm}
\title{Veille technologique : Les nouveautés de GitLab 18.4}
\author{Maël Mignard}
\date{\today}

\begin{document}
\maketitle


\section*{Introduction}
GitLab, plateforme DevSecOps très utilisée, continue d’innover avec sa version 18.4, en insistant sur la collaboration entre humains et intelligence artificielle (IA) pour accélérer et sécuriser le développement logiciel. \\

 Source : Feedly officiel de GitLab

\section*{Résumé des nouveautés majeures}
\begin{itemize}
  \item \textbf{Catalogue d’agents IA personnalisés (AI Catalog)} : GitLab introduit une bibliothèque d’agents IA spécialisés, créés sur-mesure pour automatiser des tâches répétitives, comme la gestion des bugs, la documentation ou la sécurité. Ces agents sont partageables et adaptables à chaque équipe.
    \item \textbf{Agentic Chat (Beta)} : Il s’agit d’un chat collaboratif qui permet de dialoguer avec des agents IA, de démarrer ou reprendre des sessions, de choisir quel modèle d’IA utiliser selon le contexte, et d’avoir une meilleure visibilité sur le travail des agents.
    \item \textbf{Knowledge Graph (Beta)} : Outil de cartographie intelligente du code, il permet de naviguer facilement dans de grands projets, de visualiser les liens entre fichiers, routes ou dépendances, et d’effectuer des recherches contextuelles avancées. Cela facilite l’onboarding, l’audit et la compréhension des impacts d’un changement.
    \item \textbf{Fix Failed Pipelines Flow} : Ce nouvel outil automatise la détection et la correction des échecs de pipelines CI/CD, en tenant compte de l’importance métier des différents workflows (par exemple, prioriser un service critique par rapport à un test nocturne).
    \item \textbf{Gouvernance et sécurité renforcées} : GitLab 18.4 permet de choisir précisément les modèles d’IA (LLM) utilisés, d’exclure des fichiers sensibles de l’accès IA, et d’auditer ces exclusions pour garantir la conformité et la confidentialité.
    \item \textbf{Extension des outils MCP} : L’intégration de nouveaux outils pour le serveur MCP élargit les possibilités d’interactions entre agents IA et GitLab, tout en respectant la sécurité et les permissions.
\end{itemize}
\clearpage
\section*{Conclusion}
GitLab 18.4 ne se limite pas à une simple mise à jour : il marque une avancée vers le développement logiciel intelligent et sécurisé, où l’IA devient un coéquipier au service des développeurs et de l’entreprise. Ces innovations visent à rendre les équipes plus efficaces, tout en gardant le contrôle et la sécurité sur l’ensemble du cycle de développement.

\end{document}
